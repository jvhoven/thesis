\newglossaryentry{bloom_filter}{
  name={bloom filter},
  description={Een datastructuur dat gebruikt wordt om efficient na te gaan of een item in een set zit.}
}

\newglossaryentry{0-confirmation_double_spending}{
  name={0-confirmation double spending},
  description={}
}

\newglossaryentry{dapps} {
  name={DApps},
  description={Distributed Applications -- Applicaties die gebruik maken van een Blockchain technologie, bijv. Ethereum, om op gedecentraliseerde wijze te interacteren met gebruikers}
}

\newglossaryentry{smart_contract} {
  name={Smart Contract},
  description={},
  plural={Smart Contracts}
}

\newglossaryentry{packet}{
  name={packet},
  description={Een encapsulatie van data dat gebruikt wordt door \acrfull{TCP} en \acrfull{UDP} implementaties.}
}

\newglossaryentry{double_spending}{
  name={double spending},
  description={Een aanval waarbij een reeds uitgegeven token nogmaals uitgegeven wordt}
}

\newglossaryentry{miner}{
  name={miner},
  description={Doorgaans een referentie naar iemand die het mining proces uitvoert},
  plural={miners}
}

\newglossaryentry{bootstrap_node} {
  name={bootstrap node},
  description={Gebruikers in het netwerk die constant actief zijn en gehanteerd worden als vast connectiepunt om het netwerk binnen te komen},
  plural={bootstrap nodes}
}

\newglossaryentry{minting} {
  name={minting},
  description={Een benaming voor de manier waarop een nieuw block gegenereerd wordt bij een Proof of Stake algoritme}
}

\newglossaryentry{account} {
  name={account},
  description={Een combinatie van public- en private keys waarbij de public key als identificatie gebruikt wordt}
}

\newglossaryentry{block_races}{
  name={block races},
  description={}
}

\newglossaryentry{slot_leader}{
  name={slot leader},
  description={}
}

\newglossaryentry{elector}{
  name={elector},
  description={},
  plural={electors}
}

\newglossaryentry{nonce} {
  name={nonce},
  description={Een 4-byte veld waarvan de waarde ingesteld wordt zodat de hash van een block een reeks van nullen bevat. De rest van de inhoud van een block staat hierdoor vast}
}

\newglossaryentry{stake}{
  name={stake},
  description={}
}

\newglossaryentry{voting_power}{
  name={voting power},
  description={}
}

\newglossaryentry{peer_list}{
  name={peer list},
  description={}
}

\newglossaryentry{wallet}{
  name={wallet},
  description={Software die alle adressen en secret keys bijhoudt. Het wordt gebruikt om tokens to versturen, ontvangen en op te slaan},
  plural={wallets}
}

\newglossaryentry{fork} {
  name={fork},
  description={},
  plural={forks}
}

\newglossaryentry{wallet_node} {
  name={wallet (node)},
  description={},
  plural={wallets}
}

\newglossaryentry{full_node} {
  name={full node},
  description={}
}

\newglossaryentry{mining_node} {
  name={mining node},
  description={},
  plural={mining nodes}
}

\newglossaryentry{selfish_mining}{
  name={selfish mining},
  description={}
}

\newglossaryentry{token}{
  name={token},
  description={},
  plural={tokens}
}

\newglossaryentry{difficulty}{
  name={difficulty},
  description={Een netwerk setting dat beïnvloed hoe moeilijk om het proof-of-work op te lossen}
}

\newglossaryentry{node}{
  name={node},
  description={},
  plural={nodes}
}

\newglossaryentry{UTXO} {
  name = {UTXO-model},
  description = {}
}

\newglossaryentry{distributed_hash_table} {
  name={Distributed Hash Table},
  description={}
}

\newglossaryentry{soft_fork} {
  name={Soft Fork},
  description={Een verandering in het Blockchain protocol die terugwaartse compatibiliteit heeft met eerdere versies van het protocol}
}

\newglossaryentry{hard_fork} {
  name={Hard Fork},
  description={Een verandering in het Blockchain protocol die een nieuwe regel in het netwerk introduceert, waardoor het protocol geen compatibiliteit heeft met eerder versies}
}

\newglossaryentry{peer} {
  name = {peer},
  plural = {peers},
  description = {}
}

\newglossaryentry{tunnel} {
  name = {tunnel},
  plural = {tunnels},
  description = {}
}

\newglossaryentry{ipv4} {
  name = {IPv4},
  description = {}
}

\newglossaryentry{ipv6} {
  name = {IPv6},
  description = {}
}

\newglossaryentry{OTAP} {
  name = {OTAP},
  description = {Best practice voor inrichting software ontwikkelstraat, waarbij er een Ontwikkelomgeving, Testomgeving, Acceptatieomgeving en Productieomgeving gehanteerd wordt}
}

\newacronym{CI}{CI}{Continuous Integration}
\newacronym{IDE}{IDE}{Integrated Development Environment}
\newacronym{JVM}{JVM}{Java Virtual Machine}
\newacronym{NAT}{NAT}{Network Address Translators}
\newacronym{DAG}{DAG}{Directed Acyclic Graph}
\newacronym{I2NP}{I2NP}{I2P Network Protocol}
\newacronym{I2P}{I2P}{The Invisible Internet Project}
\newacronym{P2P}{P2P}{Peer-to-Peer}
\newacronym{PoS}{PoS}{Proof of Stake}
\newacronym{PoSe}{PoSe}{Proof of Service}
\newacronym{DPoS}{DPoS}{Delegated Proof of Stake}
\newacronym{PoW}{PoW}{Proof of Work}
\newacronym{BFT}{BFT}{Byzantine Fault Tolerance}
\newacronym{DoS}{DoS}{Denial of Service}
\newacronym{tx}{tx}{transactie}
\newacronym{DHT}{DHT}{Distributed Hash Table}
\newacronym{TCP}{TCP}{Transmission Control Protocol}
\newacronym{UDP}{UDP}{User Datagram Protocol}