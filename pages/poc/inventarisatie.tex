\section{Inventarisatie}

Zoals besproken in het adviesrapport wordt het Kademlia protocol gebruikt om de topologie van het netwerk op te zetten. Daarnaast wordt er een consortium Blockchain opgezet, waarbij de toegang tot het netwerk via een EOS geïnspireerd permissiemodel geregeld wordt. Het versturen van de data van de Blockchain over het netwerk zal gedaan worden via \textit{data}, \textit{inv} en \textit{req} berichten, waarbij een \textit{data} bericht verstuurd wordt om bijvoorbeeld nieuwe transacties of blocks uit te wisselen, het \textit{inv} ter inventarisatie om duidelijk te krijgen of een andere deelnemer bepaalde data wilt hebben en een \textit{req} om data op te vragen.

\subsection{Peer-to-Peer}
Om bovenstaande keuzes te realiseren dient eerst de basis van de architectuur gerealiseerd te worden, namelijk het \acrfull{P2P} netwerk. Er zijn een aantal keuzes mogelijk met de protocollen die het \acrshort{P2P} netwerk ondersteund.

\paragraph{TCP/IP} het \acrfull{TCP} is het meest gebruikte protocol op het internet, het wordt namelijk gebruikt om data die benodigd is om een website te laden, te versturen. Een voordeel van \acrshort{TCP} is dat het protocol de garantie geeft dat data in de juiste volgorde ontvangen wordt. Het protocol wacht namelijk op bevestiging dat een \gls{packet} ontvangen is, alvorens een volgende \gls{packet} verstuurd word. Tevens zorgt dit ervoor dat data nooit corrupt raakt of verloren gaat.

\paragraph{UDP/IP} het \acrfull{UDP} werkt hetzelfde als \acrshort{TCP} alleen zit in dit protocol niet de controle of een \gls{packet} correct is aangekomen. \Glspl{packet} worden achter elkaar verstuurd zonder na te gaan of de ontvanger ze daadwerkelijk ontvangen heeft. Dit zorgt ervoor dat de overhead van het controleren niet aanwezig is, waardoor het sneller is als het \acrshort{TCP} protocol.

Omdat binnen een Blockchain implementatie garantie dat een transactie geregistreerd wordt zeer belangrijk is, zal er gebruik gemaakt worden van TCP/IP. De fail-safe mechaniek die in het protocol zit zal helpen om de Blockchain in een betrouwbare staat te houden.

\newpage
\subsection{Serialisatie}

Om data te versturen over het netwerk is het nodig om de entiteiten om te zetten naar een formaat dat verstuurd kan worden over \acrshort{TCP}. Een de-facto standaard hiervoor is het omzetten van een entiteit naar bytes, door een proces genaamd serialiseren. Hierbij zal er gekeken worden naar toepassingen die gebruikt kunnen worden op de \acrfull{JVM}.

\begin{enumerate}
  \item Java Serializable
  \\ De standaard manier van het implementeren van serialisatie binnen Java. Door het implementeren van een interface op een klasse is het mogelijk om een object out-of-the-box te serialiseren. Wanneer er gebruik wordt gemaakt van complexere entiteiten dient de programmeur de serialisatie zelf te implementeren.
  \item Protobuf
  \\ Protocol buffers zijn Google's programmeertaal-neutrale implementatie van serialisatie. Door het eenmalig definiëren van de structuur door middel van een proto-bestand, is het mogelijk om via de library eenvoudig complexe entiteiten om te zetten naar bytes en van bytes naar entiteit. Een voordeel van Protobuf is dat het samenwerkt met alle talen die de library ondersteund. Op dit moment zijn dat Java, Python, Objective-C en C++.
\end{enumerate}

Er zal gebruikt worden gemaakt van Protobuf om entiteiten binnen de Blockchain applicatie te serialiseren. Het voordeel van Protobuf, namelijk dat het samenwerkt met alle talen die de library ondersteund, zorgt ervoor dat het serialisatieproces toekomstbestendig blijft.