\section{Ontwikkelstraat}

Een ontwikkelstraat staat aan de basis van succesvolle softwareontwikkeling, en zorgt voor een duidelijke structuur tijdens de ontwikkeling van het Proof of Concept. Hierbij wordt er gebruik gemaakt van de \gls{OTAP} aanpak, waarbij er voor elke fase van het ontwikkeltraject een omgeving beschikbaar is. 

\subsection{Programmeertaal}

In de opdrachtformulering zoals beschreven door Quintor, te vinden in bijlage \ref{appendix:opdrachtformulering}, zijn er twee keuzes voor de programmeertaal voorgesteld, C\# of Java, waarmee het Proof of Concept gerealiseerd dient te worden. De keuze hierbij is al snel gevallen op Java, aangezien de afstudeerder voldoende kennis heeft van de semantiek van de taal, waardoor er tijdswinst behaald wordt. Tevens is de bedrijfsbegeleider een Java ontwikkelaar en is het mogelijk om hem te benaderen wanneer er Java expertise benodigd is.

\subsubsection{Kotlin}

Gezien recente ontwikkelingen in de Java wereld is er in overeenstemming met de andere afstudeerder voorgesteld om het Proof-of-Concept te realiseren in Kotlin. Kotlin is een programmeertaal ontwikkeld door Jetbrains, een bedrijf dat bekend staat om hun wijde assortiment aan \acrfull{IDE}'s. Ze zochten een nieuwe programmeertaal die een verbetering op Java zou zijn, maar nog steeds compatible is voor migratiedoeleinden. Naar aanleiding hiervan heeft Jetbrains een team opgezet dat zich bezig ging houden met het ontwikkelen van deze nieuwe programmeertaal. Deze programmeertaal is Kotlin geworden en heeft in februari 2016 een 1.0 release gehad. De programmeertaal is volledig open-source en compileert naar de \acrfull{JVM}, waardoor Java en Kotlin tegelijkertijd gebruikt kunnen worden. Dit is een belangrijk punt aangezien dit betekend dat alle libraries die beschikbaar zijn voor Java, ook gebruikt kunnen worden in Kotlin \citep{mediaan_kotlin}.

Het doel van het gebruiken van Kotlin is dan ook om de adoptiesnelheid, de werking, en de ervaring aan te tonen aan Quintor, zodat ze kunnen overwegen om deze programmeertaal in te zetten. In overleg met de bedrijfsbegeleider is dit goed bevonden.

\subsection{Versiebeheer}

Quintor maakt gebruik van GitLab voor het toepassen van versiebeheer. GitLab is een applicatie met features voor de gehele software development en DevOps lifecycle. Het is een open-source project en wordt gebruikt door meer dan 100.000 organisaties en heeft een community van 1900 developers die bijgedraagt hebben aan de ontwikkeling van de code \citep{gitlab_about}. Aangezien alle functionaliteiten voor het opzetten van de \gls{OTAP} omgeving, en ondersteuning tot virtualisatie indien nodig, aanwezig zijn, wordt er gebruik gemaakt van GitLab voor het toepassen van versiebeheer.

\newpage
\subsection{Testen}
\label{ontwikkelstraat:testen}
Hieronder worden de verschillende methoden en frameworks uitgelicht die gebruikt worden om de kwaliteit te waarborgen.

\subsubsection{\acrfull{TDD}}
TDD verwijst naar een programmeerstijl waarin drie activiteiten nauw met elkaar verweven zijn: ontwikkeling, testen (in de vorm van unit-tests) en ontwerp (in de vorm van refactoring) \citep{janzen2005test}.
Het kan worden beschreven door met de volgende stappen:

\begin{itemize}[noitemsep]
    \item Schrijf een unit-test die een aspect van het programma beschrijft
    \item Voer de test uit, deze faalt omdat het programma de functionaliteit mist
    \item Schrijf code die het eenvoudigst mogelijk de test laat slagen
    \item "Refactor" de code totdat deze voldoet aan de architectuur criteria
    \item Herhaal de stappen
\end{itemize}

Een belangrijk voordeel van \acrshort{TDD} is dat het promoot om kleine stappen te nemen bij het schrijven van software. Stel dat bijvoorbeeld een nieuwe functionaliteit wordt toevoegt, gecompileerd en getest. De kans is groot dat bestaande tests worden gebroken door defecten in de nieuwe code. Het is veel gemakkelijker om deze gebreken te vinden en op te lossen als je twee nieuwe coderegels hebt geschreven in plaats van 2000. De implicatie is dat hoe sneller je ontwikkeld en testen uitvoert, hoe aantrekkelijker het is om in kleinere  stappen te werk te gaan.

\subsubsection{\acrfull{ATDD}}
\acrshort{ATDD} is een methode waarbij het hele team samen discussieert over acceptatiecriteria met voorbeelden en deze vervolgens in een reeks concrete acceptatietests verwerkt voordat de ontwikkeling begint \citep{aggarwal2014acceptance}.

Deze acceptatietests vertegenwoordigen de requirements van de gebruiker en functioneren als een vorm van vereisten om te beschrijven hoe het systeem zal functioneren. Tevens dienen ze ook als een manier om te controleren of het systeem functioneert zoals bedoeld. 

\subsubsection{\acrfull{BDD}}
\acrshort{BDD} moet gericht zijn op het zakelijke gedrag dat de code implementeert: het 'waarom' achter de code \citep{wynne2017cucumber}. \acrshort{BDD} is een uitbreiding van \acrshort{TDD} en \acrshort{ATDD}. Net als bij \acrshort{TDD} wordt er in \acrshort{BDD} eerst de tests geschreven en daarna de applicatiecode. Het grote verschil dat te zien is:

\begin{itemize}[noitemsep]
    \item Tests zijn geschreven in duidelijke beschrijvende taal (Nederlands, Engels, etc..)
    \item Tests worden geschreven op de toepassing en zijn meer op de gebruiker gericht
    \item Aan de hand van voorbeelden om de vereisten te verduidelijken
\end{itemize}

\subsubsection*{Cucumber}
Cucumber is een test framework dat BDD ondersteunt \citep{wynne2017cucumber}. Met Cucumber kan het applicatie gedrag in duidelijke, betekenisvolle tekst definiëren met behulp van een eenvoudige grammatica die wordt gedefinieerd door Gherkin. Cucumber zelf is geschreven in Ruby, maar het kan worden gebruikt om code geschreven in Ruby of andere talen te `testen', inclusief maar niet beperkt tot Java/Kotlin, C\# en Python.

\subsubsection{JUnit}
JUnit is een eenvoudig, open source framework voor het schrijven en uitvoeren van unit-tests. Dit framework zal worden ingezet om de unit-testen te schrijven voor de applicatie. JUnit is het meest populaire test framework van dit moment en heeft hierdoor een stabiele community die snel innoveert. 

\subsubsection{Continous Integration}

Om de testen te koppelen aan het gebruikte versiebeheer waardoor testen automatisch uitgevoerd worden, wordt er gebruik gemaakt van \acrfull{CI}. Quintor maakt gebruik van Bamboo waarbij GitLab en Bamboo al op elkaar ingesteld zijn. Om geen overbodige werkzaamheden uit te voeren is er dan ook voor gekozen om gebruik te maken van de beschikbare Bamboo omgeving.

