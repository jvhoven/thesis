\section{Ontwikkelstraat}

Een ontwikkelstraat staat aan de basis van succesvolle softwareontwikkeling, en zorgt voor een duidelijke structuur tijdens de ontwikkeling van het Proof of Concept. Hierbij wordt er gebruik gemaakt van de \gls{OTAP} aanpak, waarbij er voor elke fase van het ontwikkeltraject een omgeving beschikbaar is. 

\subsection{Programmeertaal}

In de opdrachtformulering zoals beschreven door Quintor, te vinden in bijlage \ref{appendix:opdrachtformulering}, zijn er twee keuzes voor de programmeertaal voorgesteld, C\# of Java, waarmee het Proof of Concept gerealiseerd dient te worden. De keuze hierbij is al snel gevallen op Java, aangezien de afstudeerder voldoende kennis heeft van de semantiek van de taal, waardoor er tijdswinst behaald wordt. Tevens is de bedrijfsbegeleider een Java ontwikkelaar en is het mogelijk om hem te benaderen wanneer er Java expertise benodigd is.

\subsubsection{Kotlin}

Gezien recente ontwikkelingen in de Java wereld is er in overeenstemming met de andere afstudeerder voorgesteld om het Proof-of-Concept te realiseren in Kotlin. Kotlin is een programmeertaal ontwikkeld door Jetbrains, een bedrijf dat bekend staat om hun wijde assortiment aan \acrfull{IDE}'s. Ze zochten een nieuwe programmeertaal die een verbetering op Java zou zijn, maar nog steeds compatible is voor migratiedoeleinden. Naar aanleiding hiervan heeft Jetbrains een team opgezet dat zich bezig ging houden met het ontwikkelen van deze nieuwe programmeertaal. Deze programmeertaal is Kotlin geworden en heeft in februari 2016 een 1.0 release gehad. De programmeertaal is volledig open-source en compileert naar de \acrfull{JVM}, waardoor Java en Kotlin tegelijkertijd gebruikt kunnen worden. Dit is een belangrijk punt aangezien dit betekend dat alle libraries die beschikbaar zijn voor Java, ook gebruikt kunnen worden in Kotlin \citep{mediaan_kotlin}.

Het doel van het gebruiken van Kotlin is dan ook om de adoptiesnelheid, de werking, en de ervaring aan te tonen aan Quintor, zodat ze kunnen overwegen om deze programmeertaal in te zetten. In overleg met de bedrijfsbegeleider is dit goed bevonden.

\subsection{Versiebeheer}

Quintor maakt gebruik van GitLab voor het toepassen van versiebeheer. GitLab is een applicatie met features voor de gehele software development en DevOps lifecycle. Het is een open-source project en wordt gebruikt door meer dan 100.000 organisaties en heeft een community van 1900 developers die bijgedraagt hebben aan de ontwikkeling van de code \citep{gitlab_about}. Aangezien alle functionaliteiten voor het opzetten van de \gls{OTAP} omgeving, en ondersteuning tot virtualisatie indien nodig, aanwezig zijn, wordt er gebruik gemaakt van GitLab voor het toepassen van versiebeheer.

\subsection{Continous Integration}

Om de kwaliteit en werking van het Proof of Concept te waarborgen wordt er gebruik gemaakt van \acrfull{CI}. % TODO: Afmaken

\subsubsection{Code kwaliteit} % TODO: Beschrijven