\chapter{Evaluatie}

\textit{In dit hoofdstuk worden de resultaten van de afstudeeropdracht geëvalueerd. Er wordt gekeken naar de opgeleverde producten en de kwaliteit hiervan. Vervolgens wordt de gekozen aanpak besproken en de mogelijke afwijkingen van het afstudeerplan. Als laatste wordt er gekeken naar de beroepstaken die uitgevoerd zijn.}

\section{Producten}

\subsection{Plan van Aanpak}

Het plan van aanpak is te vinden in bijlage \ref{appendix:pva} en bevat een beschrijving van de aanpak zoals in het begin van het afstudeertraject is opgezet. Het is niet meegenomen in het iteratieve proces, waardoor het geen gedetailleerde beschrijving van de aanpakken bevat. Het plan van aanpak bevat tevens alle werkzaamheden die doorlopen zijn om tot het eindresultaat te komen en is essentieel geweest voor de uitvoering van het project. Het gaf me namelijk een goed beeld van wat er nodig was om het project tot een succesvol einde te brengen, zonder de focus van het project uit het oog te verliezen.

\subsection{Onderzoeksrapport}

Het onderzoeksrapport is te vinden in bijlage \ref{appendix:onderzoeksrapport} en bevat het resultaat van het onderzoek met als hoofdvraag ``Welke protocol implementaties kunnen toegepast worden om de onderdelen Distributed Network en Identity Management te realiseren voor een Blockchain implementatie?''. Ten tijde van het opzetten van het afstudeerplan was er nog geen kennis over het Blockchain domein, en ik ben dan ook zeer tevreden met de kennis die vergaard is met het gedane onderzoek. Helaas heb ik niet alle vragen kunnen behandelen die ik in eerste instantie in gedachten had, maar niettemin bevat het een goede fundering voor kennis voor de onderdelen Identity Management en Distributed Network in de Blockchain implementaties Bitcoin, EOS, Monero en Cardano. 

\newpage

\subsection{Architectuurdocument}

Het architectuurdocument is te vinden in bijlage \ref{appendix:architectuur} en bevat de ontwerpen van de te realiseren architectuur volgens de methode van \citep{kruchten19954+}. Ik vond het een duidelijke aanpak voor het benaderen van de architectuur van een systeem waarbij de meeste technieken overeen kwamen met de aangeleerde technieken vanuit school. Helaas heb ik niet de volledige architectuur kunnen implementeren, en dus ook niet kunnen testen of ik het correct opgesteld heb. Ik ben ervan overtuigd dat het een goede fundering beschrijft voor het realiseren van de segmenten Distributed Network en Identity Management.

\subsection{Proof of Concept}

Door het uitlopen van het onderzoek is er vrij weinig tijd over gebleven om het daadwerkelijke Proof of Concept te realiseren. Er is dan ook niet verder gekomen als het realiseren van het Peer to Peer netwerk met een opzet voor het versturen van berichten op de manier van Cardano. Het was in eerste instantie een ambitieuze implementatie, losstaand van het feit dat de complexiteit van de implementatie lag in de samenwerking met de onderdelen die gerealiseerd waren door Kevin Bos. Niettemin ben ik tevreden over wat ik bereikt waardoor ik toch de kern van het segment Distributed Network heb weten te realiseren.

\section{Aanpak}

Gedurende het afstudeertraject is de aanpak op Agile wijze uitgevoerd. Door de vele iteraties van zowel het vooronderzoek, de opzet van het onderzoek en het daadwerkelijke resultaat van het onderzoek is er veel tijd verloren gegaan, en zou ik in het vervolg ook niet voor een Agile aanpak kiezen bij het uitvoeren van onderzoek. Een groot valkuil waar ik mezelf op betrapte tijdens het onderzoeken van Blockchain protocollen is het feit dat ik teveel wil beschrijven en dat vervolgens ook wil uitlichten in het vooronderzoek om het concept duidelijk te maken voor de lezer.

\subsection{Onderzoek}

Het onderzoek is uitgevoerd door literatuurstudie waarbij geen toepassing bekend was. In andere afstudeerscripties komt het doorgaans voor dat er gebruik gemaakt wordt van toegepast onderzoek, waardoor het opzetten van de structuur in het gehele project nogal in de war kwam. Dit zorgde ervoor dat het onderzoeksrapport de grootste artefact van de afstudeeropdracht was en het Proof of Concept erbij kwam als bijkomstigheid. De keuzes die hiertoe geleid hebben konden in mijn ogen ook niet anders met hoe de opdracht vanuit Quintor gepresenteerd was, namelijk dat de insteek van de opdracht was om kennis op te doen van het Blockchain domein.

Daarnaast is het kiezen van EOS een fout geweest, dat oorzaak is geweest voor vele knelpunten in het onderzoeksproces. Dit is eigenlijk fout gegaan tijdens de selectie van implementaties en de bron die hiervoor gebruikt is. In het overzicht van coinmarketcap is mogelijk om alleen coins te tonen in plaats van de standaardweergave waarbij coins en tokens door elkaar heen weergegeven worden. Op het moment van schrijven is EOS een van de meest succesvolle \acrfull{ICO} op het moment, waarbij het geen eigen coin heeft en dus ook geen eigen Blockchain implementatie. Hierdoor is de beschikbare documentatie van het protocol (bijna) beperkt tot een technische whitepaper die toch wel wat steken laat vallen voor het begrijp technisch.

Doordat er gekozen is voor exploratief onderzoek zijn er veel termen gevonden waar ik nog niet bekend mee was maar die wel nodig waren om het totaalplaatje van het onderzoek te begrijpen. Hierbij heb ik de keuze gemaakt om het vooronderzoek achteraf aan te vullen met informatie die geanalyseerd was voor de segmenten Identity Management en Distributed Network. Dit heeft ervoor gezorgd dat het onderzoek meer tijd in beslag heeft genomen dan gepland. Dit probleem komt neer op een duidelijke afbakening maken voor het onderzoek, wat ik in de toekomst dan ook zeker zal doen.

\section{Beroepstaken}

De beroepstaken die uitgevoerd zoals opgegeven in het afstudeerplan, in te zien in bijlage \ref{appendix:afstudeerplan}, zijn hieronder verantwoord. De complexiteit is ingedeeld naar de beschrijving van beroepstaken zoals gepresenteerd in het document ``Beroepstaken van de opleiding Informatica -- Academie voor ICT \& Media, uitgave juni 2009''.

\begin{itemize}
  \item \textbf{Selecteren, methoden, technieken en tools.}
  
  Er zijn meerdere handelingen geweest die verantwoord kunnen worden onder deze beroepstaak. Tijdens het realiseren van het Proof of Concept zijn er zowel keuzes gemaakt op het gebied van de selectie van methoden, technieken en tools als bij het opstellen van de development workflow die gepaard ging met de realisatie. 
  
  De complexiteit van dit onderdeel komt neer op niveau 4, aangezien het zelfstandig is uitgevoerd en van voldoende complexiteit is in samenwerking met de inventarisatie van bestaande Blockchain technieken.

  \clearpage
  \item \textbf{Ontwerpen systeemdeel.}
  
  Het ontwerpen van het systeemdeel betrof het modelleren van de verschillende technologieën die uit de selectie van het adviesrapport gekomen zijn. De samenwerking tussen de technologieën dient modulair te zijn zodat elk losstaand deel vervangen kan worden. Het systeem dient tevens samen te werken met componenten die gerealiseerd zijn door een andere afstudeerder. Er is hierbij gebruik gemaakt van de ontwerpmethode 4+1 architectural view model zoals beschreven door \cite{kruchten19954+}. 
  
  De complexiteit van dit onderdeel komt neer op niveau 4, aangezien het systeem rekening dient te houden met de geïdentificeerde gevaren in het gedane onderzoek en het 4+1 architectural view model beschrijft de architectuur vanuit verschillende gezichtspunten.

  \item \textbf{Bouwen applicatie.}

  De realisatie van het Proof of Concept betreft het bouwen van een applicatie die aansluit op een ander deel van de Blockchain dat gerealiseerd is door een afstudeerder. Er wordt hierbij gebruik gemaakt van frameworks waarbij er redenatie aanwezig is voor gekozen frameworks. Er wordt gebruik gemaakt van versiebeheer dat gefaciliteerd is door Quintor, en er wordt containerization toegepast om een testomgeving te simuleren.

  De complexiteit van dit onderdeel komt neer op niveau 4, aangezien het aansluit op bestaande software en er gebruik gemaakt wordt van een ontwikkelomgeving inclusief testomgeving en versiebeheertool.

  \item \textbf{Initiëren en plannen testproces.}

  Helaas is het niet meer mogelijk geweest om de kwaliteit aan te tonen door het uitvoeren van een opgesteld testplan. Binnen het architectuurdocument is er wel rekening gehouden met criteria die gesteld is aan de implementatie in de vorm van non-functional requirements. Daarnaast is er wel inventarisatie gedaan naar de mogelijkheden met betrekking tot het testen van de applicatie. Dit is beschreven in hoofdstuk \ref{testen}.
\end{itemize}

\section{Functioneren binnen Quintor}

Vanaf dag één voelde ik mij zeer welkom binnen Quintor. Het bedrijf is erg betrokken bij zijn medewerkers en daarbij ook de afstudeerders en stagiaires. Het bijzondere aan afstuderen bij Quintor is dat je in een groep met afstudeerders komt. Dit is erg fijn omdat je dan nog steeds als studenten onder elkaar bent, en waarbij nodig hulp kunt verlenen aan de andere. Daarnaast was de begeleiding die aangeboden werd meer als voldoende en waren de medewerkers altijd in om mee te denken over problemen waar ik tegen aan liep. Het jammere aan afstuderen vind ik wel dat je geïsoleerd aan een opdracht werkt, waardoor je grotendeels op jezelf aangewezen bent en niet echt het bedrijfsleven ervaart. Niettemin doet Quintor er alles aan om je te betrekken bij het bedrijf. Dit doen zij door het organiseren van leuke evenementen zoals een uitje naar de verschillende festiviteiten die georganiseerd worden in Den Haag of een kennis avond over een van je favoriete onderwerpen. Tijdens de kennis avonden is er ook de gelegenheid om kennis te maken met de consultants die Quintor in dienst die normaal niet aanwezig zijn op kantoor.

\section{Leerpunten}

Hieronder heb ik een aantal leerpunten voor mijzelf neergezet die afkomstig zijn uit het afstudeertraject.

\begin{itemize}
  \item \textbf{Zorg ervoor dat de scope van je onderzoek duidelijk is, zodat je een afbakening kan maken. Dit zorgt ervoor dat je rekening houdt met de onvoorspelbaarheid van je onderzoek en dat waar nodig geëscaleerd kan worden.} Omdat de scope niet duidelijk was heb ik achteraf het vooronderzoek aangepast met kennis die benodigd was om de deelvragen uit het hoofdonderzoek te beantwoorden.
  \item \textbf{Begin vanaf het begin al met het beschrijven van de uitvoering van het onderzoek, zodat je niet nogmaals je eigen onderzoek hoeft door te nemen om te achterhalen hoe je te werk bent gegaan. Analyseer hierbij je gebruikte bronnen op validiteit en toepasbaarheid binnen de context.} De uitgevoerde stappen vertalen naar concrete processtappen binnen het hoofdonderzoek zijn niet duidelijk beschreven, waarbij een analyse op de gebruikte bronnen en welke informatie geaggregeerd is niet te achterhalen is.
  \item \textbf{Besteed meer tijd aan het duidelijk krijgen van je opdracht. Structueer hierbij je gesprekken en ga na wat je als resultaat verwacht. Wanneer dit niet zo is, bespreek dit met je begeleider.} Dit leerpunt komt voort uit de manier waarop de orientatie is gedaan in het begin van project. Hierbij zijn totaal andere resultaten gekomen als de specificatie van de criteria.
  \item \textbf{Stel een realistisch maar compleet Plan van Aanpak op. Schets hierbij de individuele stappen tot het eindresultaat, in plaats van diep in detail doorgaan op het uitwerken van de aanpak voor specifieke aspecten van het project.} Er is veel tijd besteed aan het opzetten van een onderzoeksaanpak die eigenlijk niet gebruikt is, en later een nieuwe revisie van is opgezet.
  
  \newpage
  \item \textbf{Leg afspraken over de opdracht vast met de opdrachtgever en laat het ondertekenen, zo kan de opdracht nooit veranderd worden halverwege de opdracht.} De insteek van de opdracht is naar mijn idee op twee plekken veranderd. Allereerst tijdens de orientatie, dat opgesteld was om duidelijkheid te krijgen over de niet gespecificeerde criteria. Het tweede moment is tijdens het advies gesprek, waarin wordt voorgesteld om twee verschillende aspecten van het systeem te ontwikkelen.
\end{itemize}

