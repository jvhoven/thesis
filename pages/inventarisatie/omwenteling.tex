\section{Omwenteling en implicaties}

Uit deze gesprekken is voortgekomen dat de toepassing van het Proof of Concept los staat van de uitvoering van het onderzoek, aangezien de toepassing alleen maar dient als bewijs dat de gerealiseerde Blockchain kern functioneert. Dit heeft ertoe geleid dat er beslist is over veranderingen die impact hebben op de insteek van de opdracht zoals het origineel opgesteld was. Hieronder zijn kort de veranderingen weergegeven.

%TODO: Omwenteling opdracht

\subsubsection{Generieke Blockchain} 
Er is besloten dat voor het Proof of Concept en het onderzoek de focus zal liggen op het creeren van een generieke Blockchain. Daarbij is ook het besluit genomen om de toepassing van het Proof of Concept bij het adviesrapport te betrekken. Zelf had ik hierbij mijn twijfels aangezien `generiek' ook gezien kan worden als een eis aan de implementatie, niet als een eis op de vorm en uitvoering van het onderzoek. 

Deze twijfel is dan ook aangekaart in de sprintreview, waarbij er besloten is dat dit deze eis inderdaad aan de implementatie gesteld is en niet aan het onderzoek. Dit heeft als gevolg dat de term `generiek' niet voorkomt in het onderzoek en het niet wenselijk om een dergelijk aspect als criteria voor de selectie van de implementaties te gebruiken.

\subsubsection{Criteria}

Er is besloten dat de criteria behandeld zal worden in het onderzoek, waardoor de grootte van het onderzoek zal toenemen. De toepassingsmogelijkheid zal hierbij geadviseerd worden in het adviesrapport. Hieruit zal beslist worden welke toepassing gerealiseerd kan worden binnen de beperkte tijd.