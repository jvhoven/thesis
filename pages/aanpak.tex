\chapter{Aanpak}

\textit{In dit hoofdstuk wordt de aanpak van de afstudeeropdracht besproken. Het beschrijft de beginsituatie zoals beschreven in het afstudeerplan.}

\section{Vooronderzoek} In het afstudeertraject wordt er met technologieën gewerkt welke onbekend zijn. Er is er dan ook voor gekozen om aan de hand van vooronderzoek kennis op te doen over het Blockchain domein. Er zal eerst onderzocht worden wat een Blockchain is waarna er ingegaan wordt op de toepassingen ervan. Vervolgens zal er worden gekeken worden naar de architectuur van de Blockchain en uit welke componenten het bestaat. Uiteindelijk zal er kennis opgedaan worden voor de onderdelen Identity Management en Distributed Network om zo een afbakening te maken van de onderdelen.

\section{Onderzoek} In de opdrachtomschrijving die aangeleverd is door Quintor zijn er geen duidelijke eisen en specificaties gesteld aan zowel de uitvoering als realisatie van de afstudeeropdracht. Dit heeft ertoe geleid det er een gesprek gehouden is met de Blockchain expert, Pim Otte, en de begeleider, Ben Ooms, over de eisen, afbakening en in welke mate de samenwerking met Kevin Bos benodigd zal zijn. Hieruit is naar voren gekomen dat er wederom geen specifieke eisen zijn en dat de afstudeerder onderzoek dient te doen naar implementaties om een zo goed mogelijk functioneel overzicht te creëren van de onderdelen die toegekend zijn. 

\paragraph{Opzet} Om een zo compleet mogelijk technische beschrijving van de werkingen van de gespecificeerde onderdelen te maken wordt er kwalitatief onderzoek uitgevoerd. Dit zal gedaan worden met behulp van deskresearch. De uitvoering van het deskresearch bestaat uit het onderzoeken van bestaande Blockchain protocollen, waarbij er indien mogelijk gebruik wordt gemaakt van wetenschappelijke literatuur.

\paragraph{Adviesrapport} Om in overeenstemming met de opdrachtgever een toepassing te kiezen voor de functionaliteiten en/of technieken die onderzocht zijn in de geselecteerde protocollen, zal er een adviesrapport opgesteld worden waarin deze technieken en/of technologieën aangeraden worden. 

\clearpage
\section{Proof of Concept}
De uitgekozen technieken zullen gerealiseerd worden in een Proof of Concept. Dit zal in samenwerking zijn met Kevin Bos, die het lokale gedeelte van de Blockchain ontwikkeld. De onderdelen dienen samen te werken tot een functionele Blockchain implementatie, waarbij er overlap zal zijn in de keuzes binnen de pakketselectie en realisatie.

\paragraph{Requirements} Er dienen criteria opgesteld te worden aan de hand van het resultaat van het onderzoek die van toepassing zijn op de realisatie van het Proof of Concept. Om te achterhalen wat de eisen en de toepassing waaraan het Proof of Concept moet voldoen zullen er informele interviews gehouden worden waarin requirements achterhaald worden.

\paragraph{Selecteren methoden} Voor het opzetten van een development workflow en de technieken die daarbij te pas komen in overeenstemming met Quintor zullen er beslissingen gemaakt worden op de manier waarop het Proof of Concept gerealiseerd gaat worden. Tevens zal hierbij gekeken worden naar de uitvoering van realisatie op bestaande implementaties.

\clearpage
\section{Planning}
Voor de uitvoering van het project is een globale planning opgezet die te vinden is in tabel \ref{planning}.
