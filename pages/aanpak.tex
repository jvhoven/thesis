\chapter{Aanpak}
\label{Aanpak}

\textit{In dit hoofdstuk wordt de totstandkoming van de aanpak voor de opdracht besproken. Het gaat uit van de beginsituatie zoals beschreven in het afstudeerplan, in te zien in bijlage \ref{appendix:afstudeerplan}. Daarnaast worden de afwijkingen besproken tegenover de originele opdracht, zoals geformuleerd in hoofstuk 3.}

\section{Inrichting}

Aangezien Quintor een groot voorstander is van het Agile werken zien zij ook graag het afstudeertraject in die vorm uitgevoerd worden. Tijdens de studie is er veel ervaring opgedaan met het Scrum framework dat het Agile principe ondersteund waardoor het een logische keuze is om de structuur van het project te bepalen. Omdat een groot gedeelte van het project bestaat uit het individueel uitvoeren van onderzoek zijn niet alle best practices overgenomen. 

De rollen binnen het scrum proces \citep{schwaber2011scrum} zijn als volgt gedefinieerd:
\begin{itemize}[noitemsep]
  \item \textbf{Scrum Master} - Ben Ooms
  \item \textbf{Product Owner} - Johan Tillema / Ben Ooms
  \item \textbf{Development Team} - Jeffrey van Hoven
\end{itemize}

Een sprint zal bestaan uit twee weken waarbij aan het eind van de sprint een demo van de huidige status van het project wordt gegeven aan de afstudeerbegeleider vanuit Quintor. Tijdens dit moment is het mogelijk om advies te krijgen over de uitvoering van de werkzaamheden of blokkades waar tegenaan gelopen wordt. Daarnaast wordt er onder de afstudeerders dagelijks een stand-up gehouden waarin zaken zoals de status van het project, welke werkzaamheden gepland staan voor de dag en of er obstakels zijn besproken worden.

\newpage
\section{Fases}

Er wordt uitgegaan van drie fases van het project: \textbf{onderzoek}, \textbf{ontwerp} en \textbf{ontwikkeling}, waarbij de fases ontwerp en ontwikkeling Agile uitgevoerd worden volgens de Scrum richtlijnen.

\subsection{Onderzoek}
\subsubsection{Vooronderzoek}

In het afstudeertraject wordt er met technologieën gewerkt welke onbekend zijn. Er is er dan ook voor gekozen om aan de hand van vooronderzoek kennis op te doen over het Blockchain domein. Er zal eerst onderzocht worden wat een Blockchain is waarna er ingegaan wordt op de toepassing van de techniek. Vervolgens zal er worden gekeken naar de architectuur van de Blockchain en uit welke componenten het bestaat. Uiteindelijk zal er kennis opgedaan worden voor de onderdelen Identity Management en Distributed Network om zo een afbakening te creëren van de onderdelen. Deze kennis zal gebruikt worden, in overleg met Quintor, om de opdracht vorm te geven en inzichten op te doen over de mogelijkheden met de opdracht. 

\paragraph{Dataverzameling}
Voor het opdoen van voorkennis zullen er gepubliceerde research papers, wiki’s en blogs gebruikt worden. Hierna zal er een selectie van Blockchain implementaties gemaakt worden die bestudeerd zullen worden in het onderzoek.
\subsection{Hoofdonderzoek}

Voor het uitvoeren van het onderzoek zal ik gebruik maken van deskresearch. Voor mijn deskresearch zullen er specifieke cases van Blockchain implementaties geanalyseerd worden op de segmenten Distributed Network en Identity Management. Hiervoor zal ik proberen een zo compleet mogelijk technische beschrijving van de werking van deze segmenten op te stellen.

In de opdrachtomschrijving die aangeleverd is door Quintor zijn er geen duidelijke eisen en specificaties gesteld aan zowel de uitvoering als realisatie van de afstudeeropdracht. Dit heeft ertoe geleid dat er een gesprek gehouden is met de Blockchain expert en de bedrijfsbegeleider over de eisen, afbakening en in welke mate de samenwerking met de andere afstudeerder benodigd zal zijn. Hieruit is naar voren gekomen dat er wederom geen specifieke eisen zijn en dat de afstudeerder onderzoek dient te doen naar implementaties om een zo goed mogelijk functioneel overzicht te creëren van de onderdelen die toegekend zijn. Omdat de missie van Quintor het vooroplopen op het gebied van IT ontwikkelingen is, is ervoor gekozen om exploratief onderzoek uit te voeren.
Uit het hoofdonderzoek zullen methoden en technieken geselecteerd worden die gerealiseerd zullen worden in een Proof of Concept. Alvorens deze gerealiseerd zal worden moet er nagedacht worden over hoe dit eruit zal komen te zien op technisch gebied. Het modelleren, implementeren en documenteren van een systeem vereist dat het systeem vanuit verschillende aspecten wordt bekeken. Er zal een keuze gemaakt worden betreft een methode voor het faciliteren van deze filosofie.

De uitgekozen technieken zullen gerealiseerd worden in een Proof of Concept. Dit zal in samenwerking zijn met de andere afstudeerder, die het lokale gedeelte van de Blockchain ontwikkeld. De onderdelen dienen samen te werken tot een functionele Blockchain implementatie, waarbij er overlap zal zijn in de keuzes binnen de pakketselectie en realisatie.

\paragraph{Requirements} Er dienen criteria opgesteld te worden aan de hand van het resultaat van het onderzoek die van toepassing zijn op de realisatie van het Proof of Concept. Om te achterhalen wat de eisen en de toepassing waaraan het Proof of Concept moet voldoen zullen er informele interviews gehouden worden waarin requirements achterhaald worden.

\paragraph{Selecteren methoden} Voor het opzetten van een development workflow en de technieken die daarbij te pas komen in overeenstemming met Quintor en de andere afstudeerder, zullen er beslissingen gemaakt worden op de manier waarop het Proof of Concept gerealiseerd gaat worden.


% \subsection{Adviesrapport}
% TODO: Wat moet hiermee gebeuren?
% Om in overeenstemming met de opdrachtgever een toepassing te kiezen voor de functionaliteiten en/of technieken die onderzocht zijn in de geselecteerde protocollen, zal er een adviesrapport opgesteld worden waarin deze technieken en/of technologieën aangeraden worden. 

\newpage
\section{Proof of Concept}

\clearpage
\section{Planning}
Voor de uitvoering van het project is een globale planning opgezet die te vinden is in tabel \ref{planning}.

\begin{table}[ht]
  \begin{tabular}{|p{6cm}|p{4cm}|p{4cm}|}
    \hline
    \textbf{Activiteit} & \textbf{Van} & \textbf{Tot} \\
    \hline
    Onderzoek & Week 3 & Week 7 \\
    \hline
    Advies & Week 8 & Week 9 \\
    \hline
    Selecteren methoden & Week 10 & Week 11 \\
    \hline
    Ontwikkelen & Week 11 & Week 15 \\
    \hline
    Testen & Week 16 & Week 17 \\
    \hline
    Overdracht & Week 17 & - \\
    \hline
  \end{tabular}
  \caption{Globale planning}
  \label{planning}
\end{table}

Waarbij de fases bestaan uit de volgende werkzaamheden:

\begin{multicols}{2}
  \begin{itemize}[noitemsep]
    \item \textbf{Orientatie}
    \begin{itemize}[noitemsep]
      \item Opstart
      \item Vooronderzoek
      \item Plan van Aanpak
      \item Onderzoeksopzet
      \item Probleemanalyse
    \end{itemize}
    \item \textbf{Onderzoek}
    \begin{itemize}[noitemsep]
      \item Onderzoek
      \item Selectie implementaties
      \item Theoretisch kader
      \item Bezoek begeleidend examinator
    \end{itemize}
    \item \textbf{Advies}
    \begin{itemize}[noitemsep]
      \item Adviesrapport
      \item Orientatie indeling
      \item Schrijven
      \item Voorleggen
    \end{itemize}
    \item \textbf{Selecteren methoden}
    \begin{itemize}
      \item Orientatie ontwikkelen Blockchain
      \item Selectie taal
      \item Ontwikkelomgeving
      \item Testen
    \end{itemize}
    \item \textbf{Testen}
    \begin{itemize}[noitemsep]
      \item Integratie
    \end{itemize}
    \item \textbf{Overdracht}
  \end{itemize}
\end{multicols}
