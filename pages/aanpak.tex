\chapter{Aanpak}
\label{Aanpak}

\textit{In dit hoofdstuk wordt de aanpak van de opdracht besproken. Het beschrijft de beginsituatie zoals beschreven in het afstudeerplan, in te zien in bijlage \ref{appendix:afstudeerplan}.}

\section{Beginsituatie}

De initiële beschrijving van de opdracht ging uit van criteria die nader gespecificeerd zou worden over aspecten van het Proof of Concept. Deze criteria zijn in de opdrachtomschrijving terug te vinden en gaan over snelheid, beveiliging en toepassingsmogelijkheden. Deze criteria zal gebruikt worden om de onderzoeksvragen op te stellen waarvan de antwoorden leiden tot de realisatie van het Proof of Concept.

De specificatie van deze criteria is de aanleiding geweest tot het houden van een serie gesprekken over waar Quintor heen wilt met de opdracht op het gebied van toepassingsmogelijkheden. Deze gesprekken zijn gehouden met de bedrijfsbegeleider en de Blockchain expert binnen Quintor waarbij er geprobeerd is een juiste toepassing te vinden die gerealiseerd kon worden binnen de beperkte tijd.

-- Stukje over uiteindelijk situatie e.g. --

\section{Projectinrichting}

Binnen de opdracht zal er Agile gewerkt worden. Omdat een groot gedeelte van het project bestaat uit het doen van onderzoek zijn niet alle best practices overgenomen. Per twee weken zal er een demo gedaan worden met de huidige status van het project waarbij het mogelijk is om feedback te ontvangen over blokkades of werkzaamheden die uitgevoerd dienen te worden. Daarnaast wordt er onder de afstudeerders een dagelijkse stand-up gehouden over de status van het project, welke werkzaamheden er gepland staan en of er obstakels zijn.

\section{Vooronderzoek} In het afstudeertraject wordt er met technologieën gewerkt welke onbekend zijn. Er is er dan ook voor gekozen om aan de hand van vooronderzoek kennis op te doen over het Blockchain domein. Er zal eerst onderzocht worden wat een Blockchain is waarna er ingegaan wordt op de toepassingen ervan. Vervolgens zal er worden gekeken naar de architectuur van de Blockchain en uit welke componenten het bestaat. Uiteindelijk zal er kennis opgedaan worden voor de onderdelen Identity Management en Distributed Network om zo een afbakening te creëren van de onderdelen. Deze kennis zal gebruikt worden, in overleg met Quintor, om de opdracht vorm te geven en inzichten op te doen over de mogelijkheden met de opdracht. 

Voor het opdoen van voorkennis zullen er gepubliceerde research papers, wiki’s en blogs gebruikt worden. Hierna zal er een selectie van Blockchain implementaties gemaakt worden die bestudeerd zullen worden in het onderzoek.

\newpage
\section{Onderzoek} In de opdrachtomschrijving die aangeleverd is door Quintor zijn er geen duidelijke eisen en specificaties gesteld aan zowel de uitvoering als realisatie van de afstudeeropdracht. Dit heeft ertoe geleid dat er een gesprek gehouden is met de Blockchain expert en de bedrijfsbegeleider over de eisen, afbakening en in welke mate de samenwerking met de andere afstudeerder benodigd zal zijn. Hieruit is naar voren gekomen dat er wederom geen specifieke eisen zijn en dat de afstudeerder onderzoek dient te doen naar implementaties om een zo goed mogelijk functioneel overzicht te creëren van de onderdelen die toegekend zijn. Omdat de missie van Quintor het vooroplopen op het gebied van IT ontwikkelingen is, is ervoor gekozen om literatuuronderzoek te doen.

\subsection{Opzet}

Om een zo compleet mogelijk technische beschrijving van de werkingen van de gespecificeerde onderdelen te maken wordt er kwalitatief onderzoek uitgevoerd. Er wordt onderzoek gedaan door het uitvoeren van deskresearch. Er zullen specifieke cases, implementaties van de Blockchain technologie, geselecteerd worden aan de hand van de criteria die gesteld is in ‘Inclusie- en exclusiecriteria’. 

\subsection{Adviesrapport}

Om in overeenstemming met de opdrachtgever een toepassing te kiezen voor de functionaliteiten en/of technieken die onderzocht zijn in de geselecteerde protocollen, zal er een adviesrapport opgesteld worden waarin deze technieken en/of technologieën aangeraden worden. 

\newpage
\section{Proof of Concept}
De uitgekozen technieken zullen gerealiseerd worden in een Proof of Concept. Dit zal in samenwerking zijn met de andere afstudeerder, die het lokale gedeelte van de Blockchain ontwikkeld. De onderdelen dienen samen te werken tot een functionele Blockchain implementatie, waarbij er overlap zal zijn in de keuzes binnen de pakketselectie en realisatie.

\paragraph{Requirements} Er dienen criteria opgesteld te worden aan de hand van het resultaat van het onderzoek die van toepassing zijn op de realisatie van het Proof of Concept. Om te achterhalen wat de eisen en de toepassing waaraan het Proof of Concept moet voldoen zullen er informele interviews gehouden worden waarin requirements achterhaald worden.

\paragraph{Selecteren methoden} Voor het opzetten van een development workflow en de technieken die daarbij te pas komen in overeenstemming met Quintor zullen er beslissingen gemaakt worden op de manier waarop het Proof of Concept gerealiseerd gaat worden. Tevens zal hierbij gekeken worden naar de uitvoering van realisatie op bestaande implementaties.

\clearpage
\section{Planning}
Voor de uitvoering van het project is een globale planning opgezet die te vinden is in tabel \ref{planning}.

\begin{table}[ht]
  \begin{tabular}{|p{6cm}|p{4cm}|p{4cm}|}
    \hline
    \textbf{Activiteit} & \textbf{Van} & \textbf{Tot} \\
    \hline
    Orientatie & Week 1 & Week 2 \\
    \hline
    Onderzoek & Week 3 & Week 7 \\
    \hline
    Advies & Week 8 & Week 9 \\
    \hline
    Selecteren methoden & Week 10 & Week 11 \\
    \hline
    Ontwikkelen & Week 11 & Week 15 \\
    \hline
    Testen & Week 16 & Week 17 \\
    \hline
    Overdracht & Week 17 & - \\
    \hline
  \end{tabular}
  \caption{Globale planning}
  \label{planning}
\end{table}

Waarbij de fases bestaan uit de volgende werkzaamheden:

\begin{multicols}{2}
  \begin{itemize}[noitemsep]
    \item \textbf{Orientatie}
    \begin{itemize}[noitemsep]
      \item Opstart
      \item Vooronderzoek
      \item Plan van Aanpak
      \item Onderzoeksopzet
      \item Probleemanalyse
    \end{itemize}
    \item \textbf{Onderzoek}
    \begin{itemize}[noitemsep]
      \item Onderzoek
      \item Selectie implementaties
      \item Theoretisch kader
      \item Bezoek begeleidend examinator
    \end{itemize}
    \item \textbf{Advies}
    \begin{itemize}[noitemsep]
      \item Adviesrapport
      \item Orientatie indeling
      \item Schrijven
      \item Voorleggen
    \end{itemize}
    \item \textbf{Selecteren methoden}
    \begin{itemize}
      \item Orientatie ontwikkelen Blockchain
      \item Selectie taal
      \item Ontwikkelomgeving
      \item Testen
    \end{itemize}
    \item \textbf{Testen}
    \begin{itemize}[noitemsep]
      \item Integratie
    \end{itemize}
    \item \textbf{Overdracht}
  \end{itemize}
\end{multicols}
