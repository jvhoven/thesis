\chapter{Aanbevelingen}

\textit{Tijdens het onderzoek is er helaas geen tijd geweest om de actualiteiten in het Blockchain domein te behandelen. Hiernaar is wel een korte inventarisatie gedaan, waardoor er verdergewerkt kan worden op de werkzaamheden zoals gepresenteerd in dit document.}

\section{Directed Acyclic Graph}

De implementaties NANO en IOTA maken gebruik van een \acrfull{DAG}, een nieuw soort architectuur die naar verluid de schaalbaarheid van Blockchain technologie dient te vergroten. Het is dan ook zeker interessant om te kijken of deze kennis van belang is voor Quintor.

\section{Bitcoin Lightning Network}

Bitcoin is al een tijd bezig met onderzoek naar verbetering van de transactie throughput. Dit netwerk is een tweede protocol laag bovenop een Blockchain implementatie die het mogelijk maakt om transacties direct door te zetten. Alhoewel het nog in de kinderschoenen staat en het nog vatbaar is voor bepaalde aanvallen zoals geïdentificeerd in dit onderzoek, is het een interessante techniek om te onderzoeken.

\section{Ethereum Casper}

Ethereum probeert van het Proof of Work consensus af te stappen, alleen willen ze hierdoor geen hard-fork veroorzaken. De eerste fase van Casper, Casper the Friendly Finality Gadget, is dan ook een hybride tussen Proof of Stake en Proof of Work die ingezet kan worden om het Ethereum netwerk te upgraden zonder een hard-fork te veroorzaken. Uiteindelijk zal Casper overgaan naar Casper the Friendly Ghost, een volledige implementatie van Proof of Stake.

\newpage
\section{EOS}

Op 1 juni wordt EOS gepubliceerd waarbij het eindelijk mogelijk is om deel te nemen aan het netwerk. Aangezien Quintor Blockchain wilt inzetten als ontwikkelingsplatform, is het interessant om deze implementatie te volgen aangezien het primair gebouwd is om ingezet te worden als ontwikkelingsplatform.

\section{\acrfull{NAT} Hole Punching}

\cite{ford2005peer} presenteert een aantal technieken om Hole Punching toe te passen in \acrshort{P2P} protocollen. In hoeverre dit gebruikt wordt in Blockchain implementaties is niet verder onderzocht waardoor deze studie waardevolle informatie kan bevatten. Tijdens het scannen van de tekst is er ook een stuk over \gls{ipv6} beschreven wat zeker interessant is voor toekomstige adoptie van het \gls{ipv6} adres.