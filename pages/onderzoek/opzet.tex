Het onderzoek dient voor het opstellen van het adviesrapport waarin protocollen worden  gepresenteerd  aan Quintor die mogelijk geïmplementeerd kunnen worden tijdens de realisatie van het Proof-of-Concept. Het betreft exploratief onderzoek waarin case-study gebruikt wordt om een gedetailleerde omschrijving van de onderdelen Identity Management en Distributed Network op te stellen van Blockchain implementaties die geselecteerd zijn in hoofdstuk \ref{selectie}. De kennis die hiermee wordt opgebouwd kan eventueel gebruikt worden in vervolgonderzoek.

In het onderzoek staat de onderstaande hoofdvraag centraal:
\begin{formal}
  Welke protocol implementaties kunnen toegepast worden om de onderdelen Distributed Network en Identity Management te realiseren voor een Blockchain implementatie?
\end{formal}

Omdat de hoofdvraag te groot is om in een keer te beantwoorden is het opgesplitst in de volgende deelvragen:

\begin{enumerate}[noitemsep]
  \item "Welke soorten gedistribueerde netwerken worden er gebruikt?"
  \item "Hoe werken de gedistribueerde netwerken en tegen welke gevaren zijn ze bestendig?"
  \item "Hoe wordt er omgegaan met de identiteit van de gebruiker?
\end{enumerate}

Op de volgende pagina's is per deelvraag behandeld wat de bijdrage van het antwoord oplevert aan de doelstelling, hoe de vraag beantwoord is en wordt er concreet de bevindingen besproken.

\clearpage
\section{Opzet}

Om te achterhalen welke protocol implementaties toegepast kunnen worden om de onderdelen Distributed Network en Identity Management te realiseren voer ik kwalitatief, exploratief onderzoek uit. Door het uitgevoerde vooronderzoek heb ik wel basiskennis over de onderdelen, maar weet ik nog niet de details. Om die reden heb ik er dan ook voor gekozen om het exploratief aan te pakken. Het gevaar hierbij is dat er teveel informatie geanalyseerd wordt waardoor het onderzoek te uitgebreid wordt.

\subsection{Dataverzameling}

De benodigde data zal ik verzamelen door het uitvoeren van deskresearch, waarbij ik de geselecteerde Blockchain protocol implementaties zal analyseren. De data die ik hiervoor nodig hebt komt vooral uit de whitepapers die beschikbaar zijn voor elke implementatie. Daarnaast worden er zo veel mogelijk academische bronnen gebruikt. Om deze, mogelijk aanvullende, academische bronnen te vinden zal ik gebruik maken van het schoolportaal en Google Scholar. Hierbij zal er gelet worden op de kwaliteit van de studie, door te achterhalen of het gebruikt wordt in andere studies en te kijken of het peer reviewed is.

\subsection{Dataomschrijving}

De Blockchain implementaties zullen in eerste instantie geselecteerd worden op de aanwezigheid van het onderdeel Identity Management. Hierbij zal ik kijken of de implementatie actief iets onderneemt voor bijvoorbeeld het verhogen van de privacy van de gebruiker. Daarnaast zal er gekeken worden naar de beschikbare hoeveelheid informatie.

\subsection{Analysemethode}

De methode die ik zal gebruiken voor het analyseren van de data is het uitvoeren van cumulatieve case-studies. Hierbij zullen er meerdere bronnen bekeken worden van een Blockchain implementatie waarna een aggregatie gemaakt wordt van de benodigde data voor het beantwoorden van de opgestelde vragen.