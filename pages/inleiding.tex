\chapter{Inleiding}

Dit verslag is geschreven in het kader van mijn afstudeeropdracht bij Quintor en dient ter beoordeling van de werkzaamheden die uitgevoerd zijn voor de bachelorstudie Informatica aan de Haagse Hogeschool. 

Door de snelle groei van het Blockchain domein heeft Quintor in 2017 in samenwerking met DUO/MinOCW, Groningen Declaration Network, Stichting ePortfolio Support, TNO en Rabobank, het Blockchain Field-lab Education gestart in Groningen. Het Blockchain-lab is opgezet om expertise en kennis uit te wisselen op regionaal, nationaal en internationaal gebied. De oprichting van het Blockchain Field-lab Education heeft er mede voor gezorgd dat Quintor meer kennis wilt opdoen over het Blockchain domein om zo inzicht te krijgen in hoe Blockchain technologie ingezet kan worden binnen vraagstukken vanuit klanten. 

In hoofdstuk 2 is de organisatie beschreven waar het afstudeertraject heeft plaatsgevonden. Vervolgens wordt in hoofdstuk 3 de opdracht gepresenteerd. In hoofdstuk 4 wordt de aanpak van de opdracht onderbouwd en in hoofdstuk 5 wordt de orientatie van de opdracht besproken. In hoofdstuk 6 worden de werkzaamheden van het vooronderzoek besproken waarin de basis van Blockchain technologie ter sprake komt. In hoofdstuk 7 wordt het hoofdonderzoek naar de segmenten Identity Management en Distributed Network in bestaande Blockchain implementaties beschreven. Hoofdstuk 8 presenteert het advies dat gegeven wordt naar aanleiding van het gedane onderzoek. Hoofdstuk 9 beschrijft de totstandkoming van het ontwerp voor de architectuur. In hoofdstuk 10 worden de keuzes die gemaakt zijn voor de indeling van het Proof of Concept beschreven en wordt er kort uitgelicht hoe de realisatie van het Peer-to-Peer netwerk is verlopen. In hoofdstuk 11 wordt er geëvalueerd over de producten, de aanpak, de geselecteerde beroepstaken, het functioneren in het bedrijf en de leerpunten die getrokken zijn uit het project. Tot slot worden er in hoofdstuk 11 aanbevelingen gegeven voor vervolgonderzoek.
