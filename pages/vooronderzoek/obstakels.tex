\section{Obstakels}

In het vooronderzoek is er weinig gevonden over Identity Management en zelf wist ik dan ook niet goed wat dit onderdeel voor functionaliteiten bevat. In eerste instantie is er gekeken naar de verschillende types van Blockchain, waarbij gebruikers van het systeem autorisatie hebben tot bepaalde acties. Om een beter beeld te schetsen en de afbakening van het onderdeel compleet te maken voor het onderzoek is er besloten om een gesprek te houden met de Blockchain Expert.

Uit dit gesprek is naar voren gekomen dat het Identity Management gedeelte gaat over hoe de Blockchain implementatie met public keys (de identiteit van een gebruiker) omgaat. Als tip werd er gegeven om te kijken naar de wallet software indien beschikbaar. Dit is software die de public- en private key beheert voor een gebruiker. Daarnaast is er ook een tip gegeven over het onderdeel Distributed Network. Om een goed beeld te krijgen van de aanvallen waar het netwerk tegen bestand is, is het handig om een threat model op te stellen.