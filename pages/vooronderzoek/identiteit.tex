\section{Identiteit}

\textit{
  In dit hoofdstuk wordt de vraag ``Waaruit bestaat het onderdeel Identity Management binnen Blockchain technologie?'' behandeld. Zoals beschreven in hoofdstuk \ref{chapter:blockchain} zijn anonimiteit en controleerbaarheid belangrijke eigenschappen van een Blockchain implementatie. Allereerst zal er beschreven worden wat identiteit inhoud binnen een Blockchain en welke mogelijke vormen van management er zijn. Het antwoord op deze vraag wordt gebruikt om zoektermen op te stellen en een afbakening te creëren voor de te onderzoeken protocollen.
}

In het vooronderzoek is er weinig gevonden over Identity Management en zelf wist ik dan ook niet goed wat dit onderdeel voor functionaliteiten bevat. In eerste instantie is er gekeken naar de verschillende types van Blockchain, waarbij gebruikers van het systeem autorisatie hebben tot bepaalde acties. Om een beter beeld te schetsen en de afbakening van het onderdeel compleet te maken voor het onderzoek is er besloten om een gesprek te houden met Pim Otte.

Uit dit gesprek is naar voren gekomen dat het Identity Management gedeelte gaat over hoe de Blockchain implementatie met public keys (de identiteit van een gebruiker) omgaat. Als tip werd er gegeven om te kijken naar de wallet software indien beschikbaar. Dit is software die de public- en private key beheert voor een gebruiker. Daarnaast is er ook een tip gegeven over het onderdeel Distributed Network. Om een goed beeld te krijgen van de aanvallen waar het netwerk tegen bestand is, is het handig om een threat model op te stellen.

Het onderdeel Identity Management beschrijft hoe de Blockchain omgaat met de identiteit van gebruiker die deel uitmaakt van het netwerk. Daarnaast is het mogelijk dat een bepaald type Blockchain meer doet dan alleen de identiteit van de gebruiker beheerd of maskeert. 

\subsection{Categorieën}

\cite{zheng2017overview} deelt Blockchain implementaties op in drie categorieën, waarin de zichtbaarheid en participatie in het consensus proces gelimiteerd.

\paragraph{Public} In een public Blockchain zijn alle transacties publiekelijk inzichtbaar en iedereen in het netwerk maakt onderdeel uit van het consensus proces. Dit wordt ook wel gezien als een permissionless Blockchain.

\paragraph{Consortium} In een consortium Blockchain is er een groep van vooraf geselecteerde nodes die deel uitmaken van het consensus proces. De consortium Blockchain wordt meestal gebruikt door meerdere organisaties en is gedeeltelijk gedecentraliseerd. Omdat bepaalde nodes geïdentificeerd dienen te worden wordt dit type Blockchain gezien als een permissioned Blockchain.

\clearpage
\paragraph{Private} In een private Blockchain worden alleen nodes van een specifieke organisatie toegelaten tot het consensus proces. Het wordt ook wel als een centraal netwerk gezien omdat het in volledige controle is van één organisatie. Omdat het hier gaat om volledige restrictie tot het Blockchain netwerk wordt dit type Blockchain gezien als een permissioned Blockchain.

In een consortium en een private Blockchain dient de gebruiker zich te identificeren aan de hand van een identiteit. Zoals eerder vermeld maakt Blockchain gebruik van een public- en private keys om de gebruiker te identificeren. Dit hanteert in zekere mate een permissie model waarbij de autorisatie van een gebruiker vastgelegd word aan de hand van de identificatie (i.e.\ de public key) die het netwerk gebruikt.

\subsection{Identificatie}

Identificatie wordt doorgaans gedaan aan de hand van een public key van een gebruiker. Een van de missies van Blockchain is totale anonimiteit, alleen zijn er een aantal problemen die volledige anonimiteit tegengaan. Om de terminologie duidelijk te maken wordt hieronder het verschil tussen pseudoniem en anoniem uitgelegd aan de hand van voorbeelden vanuit het Bitcoin protocol.

\begin{formal}
  ``Anonymity is the state of being not identifiable within a set of subjects, the anonymity set."
  \\ \cite{pfitzmann2001anonymity}.
\end{formal}

\paragraph{Pseudoniem} 

Bij een pseudoniem gaat er om een referentie naar de identiteit. 

\paragraph{Anoniem}





