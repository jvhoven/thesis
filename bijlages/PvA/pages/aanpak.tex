\chapter{Aanpak}

De uitvoering van dit project zal bestaan uit meerdere delen. Allereerst zal er een literatuuronderzoek gedaan worden naar een selectie van Blockchain implementaties. Uit dit onderzoek zal een adviesrapport komen die aangeboden zal worden aan het bedrijf. Hieruit zal een keuze gemaakt worden op de manier waarop de onderdelen gerealiseerd zullen worden. Om uiteindelijk de geselecteerde technieken te toetsen zal er een Proof of Concept ontwikkeld worden.

\section{Onderzoeksopzet}

In de afstudeeropdracht wordt er een adviesrapport opgesteld waarin advies wordt gegeven over de realisatie van het Proof of Concept dat betrekking heeft tot de implementatie van een Blockchain implementatie met de onderdelen Distributed Network en Identity Management. Door kwalitatieve methodieken toe te passen wordt er een technische beschrijving opgesteld van de verschillende onderdelen in de geselecteerde Blockchain implementaties. 

\subsection{Dataverzameling}

Er wordt onderzoek gedaan door middel van het uitvoeren van deskresearch. Er zullen specifieke cases, implementaties van de Blockchain technologie, geselecteerd worden aan de hand van de criteria die gesteld is in ‘Inclusie- en exclusiecriteria’. Voor het opdoen van voorkennis zullen er gepubliceerde research papers, wiki’s en beschikbare courses doorlopen worden. Hierna zal er een selectie van Blockchain implementaties gemaakt worden die bestudeerd zullen worden in het onderzoek.

\subsection{Dataomschrijving}

Om de scope van het onderzoek te beperken met betrekking tot de beschikbare tijd wordt er een selectie van drie Blockchain implementaties gemaakt. Om tot deze selectie te komen zal er een lijst van de top 20 cryptocurrencies opgesteld worden en onderzocht worden op de beschreven inclusie-en exclusiecriteria. 

\subsubsection{Inclusie- en exclusiecriteria}

De implementaties zijn in eerste instantie geselecteerd op de aanwezigheid van het onderdeel Identity Management. Daarnaast spelen de attributen open-source, of er een technische White paper beschikbaar is en het gebruikte consensus algoritme een rol tijdens de selectie van de vijf implementaties. Om diverse implementaties in kaart te brengen voor het uitbrengen van een zo goed mogelijk advies is het van belang dat de onderdelen Identity Management en Distributed Network op diverse wijze zijn geïmplementeerd. Hiervoor zijn onderstaande criteria vastgesteld.

\subsubsection{Hard-forks}

Een hard fork \citep{wiki:hardfork} is in essentie een aftakking van een bestaande blockchain door wijzigingen in de huidige structuur van de blockchain. Dit komt bijvoorbeeld voor als er een fout in de Blockchain ontdekt of misbruikt wordt. Aangezien de implementaties hiervan niet afwijken van de originele Blockchain worden hard forks niet meegenomen in het onderzoek.

\subsubsection{Consensus algoritme}

Een van de bepalende factoren van de inrichting van het onderdeel Distributed Network is het gebruik van het consensus algoritme. Dit bepaalt in hoe de verschillende verbonden cliënten overeenstemming krijgen over de waarheid van de blockchain \citep{medium:blockchain-fundamentals}. Om een compleet beeld te schetsen is het nodig om implementaties te selecteren met verschillende consensus algoritmes.

\subsection{Analysemethode}

Om te bepalen welke technieken gebruikt kunnen worden vanuit bestaande Blockchain implementaties zal er deskresearch uitgevoerd worden. Hierbij worden de werkingen van de onderdelen Distributed Network en Identity Management onderzocht en technisch beschreven.

\section{Adviesrapport}

Uit het onderzoek zal een adviesrapport komen over de manieren waarop de onderdelen Identity Management en Distributed Network opgesteld zijn binnen de onderzochte implementaties. Door het overzichtelijk maken van de resultaten uit het onderzoek zal het makkelijker zijn voor het bedrijf om een keuze te maken over de manier waarop de onderdelen gerealiseerd zullen worden.

\section{Proof-of-Concept}

Om de geselecteerde keuze(s) te toetsen zal er uiteindelijk een proof-of-Concept van de onderdelen Identity Management en Distributed Network gerealiseerd worden. Het is belangrijk dat de integriteit van deze onderdelen zo goed mogelijk bewaakt worden, waardoor er veel tijd besteed zal worden aan het testen van de implementaties. Om kennis op te doen voor het testen, ontwikkelen en ontwerpen van een blockchain implementatie zal er een selectie gemaakt worden van de gebruikte methoden, toegepaste technieken en benodigde tools.