\documentclass{report}
\usepackage[utf8]{inputenc}
\usepackage[dutch]{babel}  
\usepackage{multicol}
  \begin{document}
    {
      \bfseries\centering 
      Voortgangsverslag: Afstuderen Quintor Den Haag \\
      \today \\
      Jeffrey van Hoven
    \par}
    \vspace{1cm}
    
    {\centering
      In dit document wordt de voortgang besproken in het afstudeertraject van Jeffrey van Hoven bij het bedrijf Quintor in Den Haag. Het omvat werkzaamheden van 4 sprints waarin er gewerkt is aan het opstellen van het plan van aanpak, het doen van vooronderzoek, onderzoeksopzet en een start maken aan het uitvoeren van het onderzoek.
    \par}
    \begin{multicols}{2}
        \subsection*{Plan van Aanpak}
        Het opstellen van het plan van aanpak duurde iets langer als ingepland. Uiteindelijk is er veel tijd besteed aan het beschrijven van de aanpak en het scherpstellen van de probleemstelling en doelstelling. Dit is in overleg gebeurd met de begeleider vanuit Quintor, Ben Ooms, en een mede afstudeerder, Kevin Bos, die het lokale onderdeel van de Blockchain onderzoekt. Hier zijn meerdere gesprekken over gehouden en daarom is het opstellen van het plan van aanpak ook een beetje uitgelopen.

        \subsection*{Vooronderzoek}
        Gedurende de tijd die gebruikt werd om het plan van aanpak op te stellen is er ook vooronderzoek gedaan naar de onderdelen die deel uitmaken van mijn afstudeeropdracht, namelijk het Distributed Network en het Identity Management. Hierdoor is er tijdswinst geboekt bij het doen van het vooronderzoek terwijl het plan van aanpak uitliep. De beschrijving van de werkzaamheden is hierbij wel achtergelopen voor het afstudeerverslag, wat weer ingehaald is in de afgelopen weken.

        \subsection*{Onderzoeksopzet}
        Aan dit onderdeel is veel tijd besteed waardoor het uitgelopen is. Zo is er veel tijd verloren gegaan aan het opstellen van een selectiemethode voor de te onderzoeken implementaties. Daarnaast bleek het opstellen van de hoofdvraag en deelvragen redelijk lastig, aangezien de scope van het onderzoek niet is beperkt vanuit Quintor. Hier zijn ook meerdere gesprekken over geweest gedurende het project waaruit naar voren kwam dat de toepassing pas gegeven werd na het onderzoek.

        \subsection*{Conclusie}
        In het algemeen loop ik achter op de initiële planning die ik gemaakt heb. Zoals besproken tijdens het bezoek van dhr. T. Cocx bij Quintor, is er ruim de tijd genomen om het adviesrapport op te stellen. Die tijd kan gelijktijdig gebruikt worden om het onderzoek uit te voeren. Over het algemeen ben ik tevreden met de voortgang die ik gemaakt heb, en ik hoop dat de obstakels die ik tegengekomen ben tijdens het opstellen van het plan van aanpak en het onderzoek duidelijk terug te lezen zijn in mijn afstudeerverslag.
    \end{multicols}

\end{document}