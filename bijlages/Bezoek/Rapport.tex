\documentclass[notitlepage]{report}
\usepackage[utf8]{inputenc}
\usepackage[dutch]{babel}

\begin{document}
  {
    \bfseries\centering
    Bezoekverslag: Afstuderen Quintor Den Haag \\
    \today \\
    Jeffrey van Hoven
  \par}
  \vspace{1cm} 

  \section*{Verslag}
    Op 19 maart 2018 is dhr. T. Cocx langsgeweest voor het benodigde bedrijfsbezoek bij Quintor Den Haag om kennis te maken met het bedrijf en meer inzicht te krijgen in de afstudeeropdracht die uitgevoerd wordt door de student. In dit document worden de belangrijkste afspraken, leerpunten en conclusies besproken.

    \subsection*{Afspraken}
      Tijdens het bezoek is er kort verteld over de mogelijkheden tot het verkrijgen van feedback. Er werd nadruk gelegd op de tussentijdse beoordeling en dat het belangrijk is om deel te nemen aan het feedbackmoment dat beschikbaar is in de 10de week van de afstudeeropdracht. Hierbij is duidelijk verteld dat er verwacht wordt dat de student, indien hij gebruik wilt maken van het feedbackmoment, verwacht wordt om 60\% van het verslag afgerond te hebben. Het is ook aan de student om de afspraak te maken indien hij er gebruik van wilt maken.

    \subsection*{Leerpunten}
      Daarnaast is er gesproken over inhoudelijke werkzaamheden in relatie tot het verslag. Hierbij zijn een aantal punten genoemd over de formulering van het onderzoek. Er werd bijvoorbeeld gesproken over "technieken", wat een nogal vage term is en meerdere betekenissen kan hebben. Als suggestie werd er gegeven om het "protocol implementaties" te noemen.\\ \\
      Het tweede onderwerp was de stakeholder relatie met een andere afstudeerder die een onderdeel van de Blockchain gaat realiseren. Dit is totaal niet beschreven in het afstudeerverslag, maar toont wel de complexiteit van de opdracht. Er werd dan ook als tip gegeven om dit wel te beschrijven in het afstudeerverslag.

    \subsection*{Conclusies}
      Uit het gesprek is waardevolle feedback gekomen. Aangezien er veel nadruk werd gelegd op het feedback moment in de 10de week, ook al is er aangegeven dat het niet benodigd is, zal er zeker naar toegewerkt worden om die datum als een deadline neer te zetten.
\end{document}