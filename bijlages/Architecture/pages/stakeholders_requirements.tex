\chapter{Systeem stakeholders en requirements}

\section{Stakeholders}

Er zijn meerdere stakeholders die baat hebben bij de realisatie van dit project:

\paragraph{Quintor} De opdrachtgever en tevens de eigenaar van het project. De organisatie heeft baat bij het opdoen van kennis gedaan door dit project. Tevens zal het de eindgebruiker zijn van het systeem.

\paragraph{Kevin Bos} Heeft belang bij de realisatie van het onderdeel Distributed Network en Identity Management gezien de het gedeelte dat gerealiseerd wordt hem samen dient te werken met de componenten die voorgesteld zijn binnen dit document.

\section{Requirements}

\subsection{Business rules}
\begin{tabular}{|p{1.1cm}|p{11cm}|}
  \hline
  BR01 & Berichten dienen van type req(uest), inv(entory), data en auth(entication) te zijn. \\
  \hline
  BR02 & Transactietypes zijn: account -- om een account te registreren in het netwerk, data -- arbitraire data dat nog niet gedefinieerd is. \\
  \hline
\end{tabular}

\newpage
\subsection{Functional requirements}

\begin{tabular}{|p{1.1cm}|p{8cm}|p{3cm}|}
  \hline
  \textbf{Id} & \textbf{Beschrijving} & \textbf{Prioritering} \\
  \hline
  FR01 & Als gebruiker wil ik een transactie kunnen aanmaken. & Must have \\
  \hline
  FR02 & Als gebruiker wil ik mijn data kunnen synchroniseren. & Should have \\
  \hline
  FR03 & Als gebruiker wil ik connectie kunnen leggen met een deelnemer uit het Peer-to-Peer netwerk. & Must have \\
  \hline
  FR04 & Als gebruiker wil ik mijn openstaande connecties kunnen inzien. & Could have \\
  \hline
  FR05 & Als gebruiker wil ik kunnen toetreden in het Peer-to-Peer netwerk. & Must have \\
  \hline
  FR06 & Als gebruiker wil ik een block kunnen aanmaken. & Must have \\
  \hline
  FR07 & Als beheerder wil ik een gebruiker kunnen aanmaken. & Must have \\
  \hline
\end{tabular}

\subsection{Non-functional requirements}

\begin{tabular}{|p{1.1cm}|p{8cm}|p{3cm}|}
  \hline
  \textbf{Id} & \textbf{Beschrijving} & \textbf{ISO} \\
  \hline
  NFR01 & Het systeem dient om te kunnen gaan met deelnemers die de performance van het Peer-to-Peer netwerk proberen te verstoren. & Securability \\
  \hline  
  NFR02 & Het systeem dient om te kunnen gaan met het vervalsen van transacties. & Securability \\
  \hline
  NFR03 & Het systeem dient makkelijk uitgebreid te worden door de kerncomponenten modulair op te stellen. & Maintainability \\
  \hline
  NFR04 & Het systeem dient rekening te houden met protocol updates, en dient interactie met verouderde versies niet te ondersteunen. & Maintainability, Securability \\ 
  \hline
  NFR05 & Het systeem dient makkelijk ingezet te kunnen worden. & Deployment \\
  \hline
\end{tabular}