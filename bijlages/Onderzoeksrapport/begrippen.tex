\newglossaryentry{bloom_filter}{
  name={bloom filter},
  description={}
}

\newglossaryentry{0-confirmation_double_spending}{
  name={0-confirmation double spending},
  description={Vorm van \gls{double_spending} waarbij het validatieproces omzijlt wordt}
}

\newglossaryentry{double_spending}{
  name={double spending},
  description={Aanval waarbij geprobeerd wordt om reeds uitgegeven tokens nogmaals uit te geven}
}

\newglossaryentry{miner}{
  name={miner},
  description={Een deelnemer of een groep van deelnemers die verantwoordelijk zijn voor de creatie en validatie van nieuwe blocks. Miners worden doorgaans beloond voor hun services in de vorm van tokens},
  plural={miners}
}

\newglossaryentry{key_image} {
  name={key Image},
  plural={key Images},
  description={Structuur dat gebruikt om een transactie met \glspl{ring_signature} te valideren}
}

\newglossaryentry{sybil_attack} {
  name={sybil attack},
  description={Aanval op het netwerkgedeelte van een Blockchain waarbij er virtuele deelnemers gecreëerd worden op het netwerk om ze de processen te beïnvloeden}
}

\newglossaryentry{ring_signature} {
  name={ring signature},
  plural={ring signatures},
  description={Een digitale handtekening dat gebruikt wordt om een bericht mee te ondertekenen, waarbij het niet mogelijk is om terug te leiden wie het heeft ondertekend}
}

\newglossaryentry{stealth_address} {
  name={stealth address},
  description={Een eenmalig te gebruiken public-key die afgeleid wordt vanuit de \gls{view_key} en de \gls{spend_key}},
  plural={Stealth Addresses}
}

\newglossaryentry{view_key}{
  name={view key},
  description={Onderdeel van een account in Monero en wordt gebruikt om een derde partij inzicht te geven in gedane transacties}
}

\newglossaryentry{spend_key}{
  name={spend key},
  description={Onderdeel van een account in Monero en is benodigd om ADA uit te geven.}
}

\newglossaryentry{bootstrap_node} {
  name={bootstrap node},
  description={Vaststaande nodes waarvan hun adressen geregistreerd zijn in de broncode van de Blockchain en gebruikt worden om als eerste connectiepunt te functioneren},
  plural={bootstrap nodes}
}

\newglossaryentry{minting} {
  name={minting},
  description={Een benaming voor de manier waarop een nieuw block gegenereerd wordt bij een Proof of Stake algoritme}
}

\newglossaryentry{account} {
  name={account},
  description={Een combinatie van public- en private keys waarbij de public key als identificatie gebruikt wordt}
}

\newglossaryentry{block_races}{
  name={block races},
  description={Hiermee wordt de onderlinge competitie bedoeld van \glspl{miner}, waarbij ze ``racen'' om als eerste een block te produceren}
}

\newglossaryentry{slot_leader}{
  name={slot leader},
  description={Block producer die verantwoordelijk is voor het creëren van een block},
  plural={slot leaders}
}

\newglossaryentry{elector}{
  name={elector},
  description={Groep van \glspl{node} die verantwoordelijk is voor het verkiezen van  \glspl{slot_leader}},
  plural={electors}
}

\newglossaryentry{nonce} {
  name={nonce},
  description={Een 4-byte veld waarvan de waarde ingesteld wordt zodat de hash van een block een reeks van nullen bevat. De rest van de inhoud van een block staat hierdoor vast}
}

\newglossaryentry{stake}{
  name={stake},
  description={Investering in de Blockchain proportioneel naar het type consensus, meestal gebruikt in \acrshort{PoS} implementaties}
}

\newglossaryentry{voting_power}{
  name={voting power},
  description={Hoeveel zeggenschap een \gls{node} heeft in het netwerk gebaseerd op attributen als hashing power en \gls{stake}}
}

\newglossaryentry{peer_list}{
  name={peer list},
  description={Lijst van \glspl{peer} waarmee connectie is gemaakt}
}

\newglossaryentry{wallet}{
  name={wallet},
  description={Software die alle adressen en secret keys bijhoudt. Het wordt gebruikt om tokens to versturen, ontvangen en op te slaan},
  plural={wallets}
}

\newglossaryentry{fork} {
  name={fork},
  description={Splitsing in het netwerk dat veroorzaakt is door een kleine wijziging in het protocol},
  plural={forks}
}

\newglossaryentry{wallet_node} {
  name={wallet (node)},
  description={\Gls{node} die een gereduceerde staat van het Blockchain bevat, waarin alleen de transacties opgenomen worden die betrekking hebben op de public- en private key combinatie},
  plural={wallets}
}

\newglossaryentry{full_node} {
  name={full node},
  description={\Gls{node} die alle functionaliteit kan uitvoeren die de Blockchain implementatie aanbiedt}
}

\newglossaryentry{mining_node} {
  name={mining node},
  description={\Gls{node} als enige taak heeft om de rol van \gls{miner} te vervullen},
  plural={mining nodes}
}

\newglossaryentry{selfish_mining}{
  name={selfish mining},
  description={Aanval waarbij er door een kwaadwillende \gls{mining_node} blocks achtergehouden worden}
}

\newglossaryentry{token}{
  name={token},
  description={Abstracte term voor data die verstuurd wordt over het netwerk van de Blockchain, meestal wordt hiermee een cryptocurrency bedoeld},
  plural={tokens}
}

\newglossaryentry{difficulty}{
  name={difficulty},
  description={Een netwerk instelling dat beïnvloed hoe moeilijk het proof-of-work op te lossen is door het aanpassen van de \gls{nonce}}
}

\newglossaryentry{node}{
  name={node},
  description={Computer dat in verbinding staat met het netwerk van de Blockchain},
  plural={nodes}
}

\newglossaryentry{UTXO} {
  name = {UTXO-model},
  description = {Gegevenstructuur voor transacties waarbij een transactie bestaat uit een lijst van inputs en outputs}
}

\newglossaryentry{distributed_hash_table} {
  name={Distributed Hash Table},
  description={Gegevenstructuur dat gebruikt wordt in gedecentralizeerde systemen waarin efficiënte data opgezocht kan worden dat zich bevind bij een van de \glspl{peer} binnen het netwerk}
}

\newglossaryentry{soft_fork} {
  name={Soft Fork},
  description={Een verandering in het Blockchain protocol die terugwaartse compatibiliteit heeft met eerdere versies van het protocol}
}

\newglossaryentry{hard_fork} {
  name={Hard Fork},
  description={Een verandering in het Blockchain protocol die een nieuwe regel in het netwerk introduceert, waardoor het protocol geen compatibiliteit heeft met eerder versies}
}

\newglossaryentry{peer} {
  name = {peer},
  plural = {peers},
  description = {Synoniem voor \gls{node}}
}

\newglossaryentry{tunnel} {
  name = {tunnel},
  plural = {tunnels},
  description = {}
}

\newacronym{I2NP}{I2NP}{I2P Network Protocol}
\newacronym{I2P}{I2P}{The Invisible Internet Project}
\newacronym{P2P}{P2P}{Peer-to-Peer}
\newacronym{PoS}{PoS}{Proof of Stake}
\newacronym{PoSe}{PoSe}{Proof of Service}
\newacronym{DPoS}{DPoS}{Delegated Proof of Stake}
\newacronym{PoW}{PoW}{Proof of Work}
\newacronym{BFT}{BFT}{Byzantine Fault Tolerance}
\newacronym{DoS}{DoS}{Denial of Service}
\newacronym{tx}{tx}{transactie}
\newacronym{DHT}{DHT}{Distributed Hash Table}