

Er zijn drie onderdelen van het Bitcoin systeem die interessant zijn voor het analyseren van het systeem in relatie tot de identiteit van de gebruiker. Ten eerste is de gehele historie van Bitcoin transacties publiekelijk in te zien. Zoals eerder vermeld is dit nodig om zonder centrale autoriteit validatie van de transacties te doen. Het tweede is het UTXO-model dat gebruikt wordt om uitgaves en inkomsten bij te houden. In dit model bestaat een transactie uit meerdere inputs en outputs, waarbij de input een eerdere output van een transactie is geweest. Ten derde zijn de betaler en de ontvanger van een transactie gekoppeld aan de transactie door middel van een public key. 

\cite{reid2013analysis} stelt dat deze drie onderdelen, met name de publieke toegankelijkheid van de Bitcoin transacties en de input-output relatie tussen transacties en public-keys, ingedeeld kunnen worden in twee verschillende netwerken die tesamen opereren, het \textit{transaction network} en het \textit{user network}. Waarbij het \textit{transaction network} de stroom van Bitcoins beschrijft tussen transacties over de tijd, en het \textit{user network} tussen gebruikers over de tijd. Door het analyseren van de structuur van deze twee netwerken aan de hand van de informatie uit het Bitcoin netwerk, is er geconcludeerd dat het mogelijk is om verschillende public keys met elkaar te associëren, en het met de juiste middelen het mogelijk is om de activiteit van een gebruiker gedetailleerd in kaart te brengen.

Hierbij voldoet het bitcoin protocol met name niet aan de de untraceability eis. Alle transacties die gedaan worden tussen de deelnemers van het netwerk zijn publiekelijk in te zien en elke transactie kan herleid worden naar de verstuurder en ontvanger. Ook is het indirect mogelijk om twee uitgaande transacties naar dezelfde persoon aan te tonen binnen het netwerk.
