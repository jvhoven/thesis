\section{Functionaliteit}

\textit{In dit hoofdstuk wordt er onderzocht welke functionaliteiten de netwerken ondersteunen die uitgelicht zijn in het vorige hoofdstuk. Er wordt gekeken welke gevaren er kunnen optreden in het netwerk aan de hand van het opstellen van een threat model.}

\paragraph{Majority Attack}
\cite{nakamoto2008bitcoin} stelt dat het uitvoeren van een majority attack op het netwerk onpraktisch is omdat het uitvoeren ervan niet opweegt tegen de kosten voor de benodigde hardware om de rekenkracht te behalen die hiervoor nodig is. Dit blijkt niet altijd het geval, \cite{eyal2014majority} beschrijft namelijk een strategie genaamd Selfish Mining waarbij er gevalideerde blocks achtergehouden worden voor het netwerk waardoor er opzettelijk een fork wordt gecreëerd. De eerlijke miners zullen verder werken aan de publiekelijke blockchain terwijl de uitvoerder van het Selfish Mining strategie verder werkt op de achtergehouden blockchain. Als de uitvoerder meer blokken ontdekt onstaat er een voorsprong op de publiekelijke blockchain en worden de blocks nog steeds achtergehouden. Wanneer de lengte van de publiekelijke blockchain de lengte van de achtergehouden blockchain benaderd, zal de uitvoerder de blockchain publiceren. Dit leid ertoe dat miners die het Bitcoin protocol volgen hun middelen verspillen aan het minen van cryptopuzzles die er niet toe doen.

\paragraph{Denial of Service} 
Over de jaren heen zijn er kwetsbaarheden in het Bitcoin protocol geïdentificeerd die het mogelijk maken om een \acrshort{DoS} aanval uit te voeren. De meest recente\footnote{Er zijn recentere aanvallen op het Bitcoin protocol geweest waarbij er \acrshort{DoS} aanval heeft plaatsgevonden maar deze zijn niet nader gespecificeerd, zie: \href{https://en.bitcoin.it/wiki/Common_Vulnerabilities_and_Exposures}{"Common Vulnerabilities and Exposures - Bitcoin Wiki"}.} aanval {\citep{nist_bitcoind_dos} exploiteert een zwakheid in de implementatie van een \Gls{bloom_filter}, een filter die onder andere gebruikt wordt door \glspl{wallet_node} om alleen transacties binnen te halen waarbij de deelnemer betrokken is. Hierdoor was het mogelijk om een sequentie van berichten te sturen die ervoor zorgde dat een volledige node binnen het netwerk overbelast werd.

\paragraph{Eclipse Attack} 
\citeauthor{heilman2015eclipse} heeft aangetoond dat Bitcoin's peer discovery mechanisme toegankelijk is voor een \textit{Eclipse attack}. Door de manier waarop het Peer Discovery mechanisme werkt is het mogelijk om de lijst van connecties zo te manipuleren dat nieuwe deelnemers doorgestuurd worden naar kwaadwillende deelnemers.

\paragraph{Double spending}
\citeauthor{karame2012double} toont aan dat het in het beginstadium van het Bitcoin protocol mogelijk was om via zogenaamde 'fast payments' een double spending aanval uit te voeren. 

\clearpage
De cardano implementatie is een public Blockchain waarbij alle transacties inzichtbaar zijn en iedereen mee kan doen aan het consensus proces. Het maakt gebruik van public- en private keys om pseudonimiteit te waarborgen, waarbij de elliptic curve cryptografie implementatie Curve25519 toegepast wordt om de public- en private key te genereren. Binnen Cardano worden er verschillende adressen gebruikt om transacties van een bestemming te voorzien \citep[''Addresses in Cardano SL'']{cardano_wiki}:

\begin{enumerate}
  \item \textbf{public key address}
  \\ Een base58 gecodeerde string van de public key dat gebruikt wordt als bestemming van een transactie.
  \item \textbf{script address}
  \\ Wordt gebruikt voor het Pay to Script Hash principe, waarbij er in plaats van de public key gebruikt wordt als bestemming, een validatie script verstuurd wordt die gebruik maakt van een zogenaamde redemption script. Om de waarde van de transactie te claimen dient het validatie script positief uit te vallen.
  \item \textbf{redeem address}
  \\ Wordt gebruikt voor het Pay to Public Key Hash principe, waarbij er een hash gecreëerd wordt wat ervoor zorgt dat de public key alleen publiekelijk geregistreerd wordt wanneer de output van een transactie wordt uitgegeven.
\end{enumerate}




\clearpage
EOS is een consortium Blockchain waarin de identiteit van een gebruiker vastgelegd wordt in een account model, waarbij een account identificeerbaar is door een unieke naam van maximaal twaalf karakters. Handeling zijn geresticteerd door middel van een Role Based Permissie systeem. Om dit mogelijk te maken dient een gebruiker allereerst geautoriseerd te zijn alvorens deel te kunnen nemen aan het netwerk. Centraal in de implementatie staat de notie van Actions \& Handlers. Elk account (i.e.\ deelnemer) heeft een eigen database die alleen toegankelijk is door gedefinieerde action handlers. Dit systeem is soortgelijk aan smart contracts zoals in gebruik bij Ethereum.


\clearpage
\paragraph{Double Spending} door gebruik te maken van \glspl{ring_signature} wordt de herkomst van een transactie gemaskeerd door de handtekening van de verstuurder te groeperen met handtekeningen vanuit outputs die reeds gedaan zijn in de Blockchain. Een probleem dat hierbij optreed is de mogelijkheid tot de uitvoering van \gls{double_spending} omdat een transactie lastiger is te valideren. Hierdoor maakt Monero gebruik van \gls{key_image}. Een \gls{key_image} wordt gebruikt om te valideren dat de private key die gebruikt is om de transactie te ondertekenen niet eerder gebruikt is, zonder te onthullen welke handtekening het is.


\clearpage
\section{Gevaren}



\clearpage

% \begin{centering}
%   \begin{table}
%     \makebox[\textwidth]{%
%       \begin{tabular}{@{}l|cccccc@{}}
%         \toprule
%         \textbf{Implementatie} & \textit{Eclipse Attack} & \textit{Majority Attack} & \textit{Sybil Attack} & \textit{Denial of Service} & \textit{Double Spending} & \textit{Nothing at Stake} \\ 
%         \midrule
%         \textit{Bitcoin} & \Rmnum{3} & \Rmnum{3} & \Rmnum{2} & \Rmnum{1} & \Rmnum{2} & ... \\
%         \textit{Cardano} & \Rmnum{5} & \Rmnum{3} & \Rmnum{3} & \Rmnum{2} & \Rmnum{4} & ... \\
%         \textit{EOS} & - &  &  &  &  &  \\
%         \textit{Monero} &  &  &  & ... &  & ... \\
%         \bottomrule
%         \multicolumn{6}{l}{} \Rmnum{1} meerdere keren uitgevoerd, \Rmnum{2} uitgevoerd, \Rmnum{3} theoretisch mogelijk, \Rmnum{4} onbekend, \Rmnum{5} niet mogelijk & \\
%       \end{tabular}
%     }
%     \caption{Indicatie van gevaren in Blockchain implementaties.}
%     \label{my-label}
%   \end{table}
% \end{centering}

\paragraph{Majority Attack}
\cite{nakamoto2008bitcoin} stelt dat het uitvoeren van een majority attack op het netwerk onpraktisch is omdat het uitvoeren ervan niet opweegt tegen de kosten voor de benodigde hardware om de rekenkracht te behalen die hiervoor nodig is. Dit blijkt niet altijd het geval, \cite{eyal2014majority} beschrijft namelijk een strategie genaamd Selfish Mining waarbij er gevalideerde blocks achtergehouden worden voor het netwerk waardoor er opzettelijk een fork wordt gecreëerd. De eerlijke miners zullen verder werken aan de publiekelijke blockchain terwijl de uitvoerder van het Selfish Mining strategie verder werkt op de achtergehouden blockchain. Als de uitvoerder meer blokken ontdekt onstaat er een voorsprong op de publiekelijke blockchain en worden de blocks nog steeds achtergehouden. Wanneer de lengte van de publiekelijke blockchain de lengte van de achtergehouden blockchain benaderd, zal de uitvoerder de blockchain publiceren. Dit leid ertoe dat miners die het Bitcoin protocol volgen hun middelen verspillen aan het minen van cryptopuzzles die er niet toe doen.

\paragraph{Denial of Service} 
Over de jaren heen zijn er kwetsbaarheden in het Bitcoin protocol geïdentificeerd die het mogelijk maken om een \acrshort{DoS} aanval uit te voeren. De meest recente\footnote{Er zijn recentere aanvallen op het Bitcoin protocol geweest waarbij er \acrshort{DoS} aanval heeft plaatsgevonden maar deze zijn niet nader gespecificeerd, zie: \href{https://en.bitcoin.it/wiki/Common_Vulnerabilities_and_Exposures}{"Common Vulnerabilities and Exposures - Bitcoin Wiki"}.} aanval {\citep{nist_bitcoind_dos} exploiteert een zwakheid in de implementatie van een \Gls{bloom_filter}, een filter die onder andere gebruikt wordt door \glspl{wallet_node} om alleen transacties binnen te halen waarbij de deelnemer betrokken is. Hierdoor was het mogelijk om een sequentie van berichten te sturen die ervoor zorgde dat een volledige node binnen het netwerk overbelast werd.

\paragraph{Eclipse Attack} 
\citeauthor{heilman2015eclipse} heeft aangetoond dat Bitcoin's peer discovery mechanisme toegankelijk is voor een \textit{Eclipse attack}. Door de manier waarop het Peer Discovery mechanisme werkt is het mogelijk om de lijst van connecties zo te manipuleren dat nieuwe deelnemers doorgestuurd worden naar kwaadwillende deelnemers.

\paragraph{Double spending}
\citeauthor{karame2012double} toont aan dat het in het beginstadium van het Bitcoin protocol mogelijk was om via zogenaamde 'fast payments' een double spending aanval uit te voeren. 

\newpage

De cardano implementatie is een public Blockchain waarbij alle transacties inzichtbaar zijn en iedereen mee kan doen aan het consensus proces. Het maakt gebruik van public- en private keys om pseudonimiteit te waarborgen, waarbij de elliptic curve cryptografie implementatie Curve25519 toegepast wordt om de public- en private key te genereren. Binnen Cardano worden er verschillende adressen gebruikt om transacties van een bestemming te voorzien \citep[''Addresses in Cardano SL'']{cardano_wiki}:

\begin{enumerate}
  \item \textbf{public key address}
  \\ Een base58 gecodeerde string van de public key dat gebruikt wordt als bestemming van een transactie.
  \item \textbf{script address}
  \\ Wordt gebruikt voor het Pay to Script Hash principe, waarbij er in plaats van de public key gebruikt wordt als bestemming, een validatie script verstuurd wordt die gebruik maakt van een zogenaamde redemption script. Om de waarde van de transactie te claimen dient het validatie script positief uit te vallen.
  \item \textbf{redeem address}
  \\ Wordt gebruikt voor het Pay to Public Key Hash principe, waarbij er een hash gecreëerd wordt wat ervoor zorgt dat de public key alleen publiekelijk geregistreerd wordt wanneer de output van een transactie wordt uitgegeven.
\end{enumerate}




\newpage

EOS is een consortium Blockchain waarin de identiteit van een gebruiker vastgelegd wordt in een account model, waarbij een account identificeerbaar is door een unieke naam van maximaal twaalf karakters. Handeling zijn geresticteerd door middel van een Role Based Permissie systeem. Om dit mogelijk te maken dient een gebruiker allereerst geautoriseerd te zijn alvorens deel te kunnen nemen aan het netwerk. Centraal in de implementatie staat de notie van Actions \& Handlers. Elk account (i.e.\ deelnemer) heeft een eigen database die alleen toegankelijk is door gedefinieerde action handlers. Dit systeem is soortgelijk aan smart contracts zoals in gebruik bij Ethereum.


\newpage

\paragraph{Double Spending} door gebruik te maken van \glspl{ring_signature} wordt de herkomst van een transactie gemaskeerd door de handtekening van de verstuurder te groeperen met handtekeningen vanuit outputs die reeds gedaan zijn in de Blockchain. Een probleem dat hierbij optreed is de mogelijkheid tot de uitvoering van \gls{double_spending} omdat een transactie lastiger is te valideren. Hierdoor maakt Monero gebruik van \gls{key_image}. Een \gls{key_image} wordt gebruikt om te valideren dat de private key die gebruikt is om de transactie te ondertekenen niet eerder gebruikt is, zonder te onthullen welke handtekening het is.

\newpage

