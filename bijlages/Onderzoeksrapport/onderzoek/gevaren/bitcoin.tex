\paragraph{Majority Attack}
\cite{nakamoto2008bitcoin} stelt dat het uitvoeren van een majority attack op het netwerk onpraktisch is omdat het uitvoeren ervan niet opweegt tegen de kosten voor de benodigde hardware om de rekenkracht te behalen die hiervoor nodig is. Dit blijkt niet altijd het geval, \cite{eyal2014majority} beschrijft namelijk een strategie genaamd Selfish Mining waarbij er gevalideerde blocks achtergehouden worden voor het netwerk waardoor er opzettelijk een fork wordt gecreëerd. De eerlijke miners zullen verder werken aan de publiekelijke blockchain terwijl de uitvoerder van het Selfish Mining strategie verder werkt op de achtergehouden blockchain. Als de uitvoerder meer blokken ontdekt onstaat er een voorsprong op de publiekelijke blockchain en worden de blocks nog steeds achtergehouden. Wanneer de lengte van de publiekelijke blockchain de lengte van de achtergehouden blockchain benaderd, zal de uitvoerder de blockchain publiceren. Dit leid ertoe dat miners die het Bitcoin protocol volgen hun middelen verspillen aan het minen van cryptopuzzles die er niet toe doen.

\paragraph{Denial of Service} 
Over de jaren heen zijn er kwetsbaarheden in het Bitcoin protocol geïdentificeerd die het mogelijk maken om een \acrshort{DoS} aanval uit te voeren. De meest recente\footnote{Er zijn recentere aanvallen op het Bitcoin protocol geweest waarbij er \acrshort{DoS} aanval heeft plaatsgevonden maar deze zijn niet nader gespecificeerd, zie: \href{https://en.bitcoin.it/wiki/Common_Vulnerabilities_and_Exposures}{"Common Vulnerabilities and Exposures - Bitcoin Wiki"}.} aanval {\citep{nist_bitcoind_dos} exploiteert een zwakheid in de implementatie van een \Gls{bloom_filter}, een filter die onder andere gebruikt wordt door \glspl{wallet_node} om alleen transacties binnen te halen waarbij de deelnemer betrokken is. Hierdoor was het mogelijk om een sequentie van berichten te sturen die ervoor zorgde dat een volledige node binnen het netwerk overbelast werd.

\paragraph{Eclipse Attack} 
\citeauthor{heilman2015eclipse} heeft aangetoond dat Bitcoin's peer discovery mechanisme toegankelijk is voor een \textit{Eclipse attack}. Door de manier waarop het Peer Discovery mechanisme werkt is het mogelijk om de lijst van connecties zo te manipuleren dat nieuwe deelnemers doorgestuurd worden naar kwaadwillende deelnemers.

\paragraph{Double spending}
\citeauthor{karame2012double} toont aan dat het in het beginstadium van het Bitcoin protocol mogelijk was om via zogenaamde 'fast payments' een double spending aanval uit te voeren. 
