\section{Identiteit van de gebruiker}
\textit{In dit hoofdstuk wordt er onderzocht hoe de implementaties omgaan met de identiteit van de gebruiker die interacteert met de Blockchain implementatie.}

Blockchain kan een zeker mate van privacy garanderen door de public en private keys, wat ervoor zorgt dat een gebruiker niet zijn echte identiteit hoeft te hanteren om met het systeem te interacteren. Echter, \cite{meiklejohn2013fistful} toont aan dat blockchain niet de transactionele privacy kan waarborgen omdat de waarden van alle transacties en saldo van elke public key openbaar inzichtbaar zijn.

\citet{Okamoto:1991:UEC:646756.705374} beschrijft zes criteria waaraan de ideale implementatie van elektronisch geld moet voldoen. In het bijzonder worden er twee criteria genoemd:

\begin{itemize}
  \item \textbf{Untraceability:} voor elke inkomende transactie hebben alle mogelijke afzenders gelijke kansen om geïdentificeerd te worden als verstuurder. 
  
  \item \textbf{Unlinkability:} voor elke twee uitgaande transacties moet het onmogelijk zijn om aan te tonen dat ze naar dezelfde persoon verstuurd zijn.
\end{itemize}

Zie papers: Recent Development in Blockchain

\subsubsection{Bitcoin}


\subsubsection{EOS}

In EOS alle accounts kunnen gecreëerd worden met een unieke gebruikersnaam.

\subsubsection{Monero}

Monero probeert de untraceability criteria op te lossen door het gebruik van \textbf{spend keys} en \textbf{view keys}.

\paragraph{Spend} keys 

\paragraph{View} keys