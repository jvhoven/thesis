\newpage
\chapter{Resultaten}

\section{Soorten netwerken}

\textit{In dit hoofdstuk wordt er onderzocht welke verschillende netwerken er gebruikt worden in bestaande implementaties. Hierbij wordt zowel de definitie van soorten en de selectie van implementaties gebruikt uit de resultaten van het vooronderzoek.}

"De distributie van informatie en het probleem van wederzijdse overeenstemming over een consistente staat van het netwerk vormt een uitdaging, zeker in de aanwezigheid van zelfzuchtige en/of kwaadwillende deelnemers" - \citet{7423672}. Het is een uitdaging die bekend staat als het Byzantine Generals' Problem, en is beschreven door \citet{lamport1982byzantine}. Het stelt dat het essentieel is voor een betrouwbaar computersysteem om te kunnen gaan met fouten die optreden in een of meer van de componenten, waardoor het kan voorkomen dat er conflicterende informatie verstuurd wordt naar de andere componenten van het systeem. In hoeverre een computersysteem hiermee om kan gaan wordt de \acrfull{BFT} genoemd en wordt aangeduid als: $ f = [\frac{N - 1}{t}] $ waarbij \(N\) componenten van een computersysteem zijn en \(t\) de foutieve componenten.

In blockchain implementaties zijn de componenten die onbetrouwbaar zijn de deelnemers van het peer-to-peer netwerk. Het soort netwerk is dan ook verbonden met de manier waarop consensus bereikt wordt tussen de deelnemers van het netwerk en is getypeerd als het consensus protocol dat geïmplementeerd is.

\newpage

\subsection{Proof of Work}
\label{chapter-proof-of-work}
De originele implementatie van Blockchain technologie is gepresenteerd door \citet{nakamoto2008bitcoin} in \textit{"Bitcoin: A peer-to-peer electronic cash system"}. Het maakt gebruik van een algoritme genaamd \acrfull{PoW} om consensus te bereiken. Hierbij gaat het om het oplossen van een wiskundig probleem $Y \in \mathbb{N} < f(X + n)$ waarbij $f$ een hash functie is, $n$ de \gls{nonce}, $X$ de data en $Y$ de \gls{difficulty}.

\begin{wrapfigure}{r}{0.5\textwidth}
  \includegraphics[width=0.5\textwidth, height=25mm]{proof-of-work}
  \caption[Proof-of-Work in Bitcoin]{Werking Proof-of-Work, van \citet{ProofofWork}. Wanneer de eerste vier bits ($Y = 4$) van de hash 0 zijn is de proef opgelost. }
  \label{proof-of-work}    
\end{wrapfigure}
In het geval van Bitcoin is de $Y$ waarde een getal die aanduid wat de \gls{difficulty} is om de hash te berekenen en wordt de $X$ waarde incrementeel opgehoogd. Een voorbeeld is gegeven in fig. \ref{proof-of-work}. Dit proces zorgt ervoor dat de integriteit van de data in een block op de Blockchain bewaakt wordt. Wanneer een kwaadwillende deelnemer aan het netwerk de data van een block wilt aanpassen die reeds opgenomen is in de Blockchain, kan er via het \acrshort{PoW} makkelijk gevalideerd worden of het block invalide is.

Daarnaast beschrijft de bedenker van het protocol, Satoshi Nakamoto, het \acrshort{PoW} algoritme als 'one-CPU-one-vote'. Aangezien het gebruikte hashing algoritme geen limitaties stelt tot de zogeheten \gls{voting_power} van een deelnemer in het netwerk creëert het gunstige omstandigheden voor high-end GPU eigenaren tegenover high-end CPU eigenaren \citep[p.~2]{van2013cryptonote}.

\textbf{Monero} maakt gebruik van het CryptoNight algoritme \citep{noether2014monero}, een implementatie gebaseerd op CryptoNote, waarin gebruik gemaakt wordt van een egalitair Proof of Work \citep[p.~11]{van2013cryptonote}. In contrast met het Bitcoin protocol Proof of Work algoritme is het ontworpen om inefficiënt berekenbaar te zijn op een GPU, waardoor er gelijke kansen zijn voor de deelnemers van het netwerk die het mining proces uitvoeren.

\newpage
\subsection{Proof of Stake}

"Een eerste overweging met betrekking tot de werking van blockchain protocollen gebaseerd op Proof of Work -- zoals Bitcoin -- is de energie benodigd voor hun uitvoering." - \citet{kiayias2017ouroboros}.
In een onderzoek gedaan door \citeauthor{ODwyer:Bitcoin} in \citeyear{ODwyer:Bitcoin} naar het energieverbruik van het Bitcoin mining netwerk is geschat dat onder redelijke omstandigheden het netwerk gelijk stond met het energiegebruik van Ierland. Om deze reden zijn er onderzoeken en experimenten gedaan naar alternatieve consensus algoritmes. \acrfull{PoS} is een consensus algoritme waarbij, in plaats van het verspillen van elektriciteit om zware rekenkundige problemen op te lossen, een deelnemer geselecteerd wordt om het volgende blok te genereren (doorgaans \gls{minting} genoemd) op basis van willekeurige selectie en rijkdom of leeftijd (i.e., de stake).

\paragraph{Cardano} maakt gebruik van \acrshort{PoS} waarbij iedere deelnemer van het netwerk met een positief balans (e.g. stake) als stakeholders gezien worden. Om uitgekozen te worden om een nieuw block te genereren moet een stakeholder geselecteerd worden als \gls{slot_leader}. De implementatie verdeelt de fysieke tijd in tijdvakken en elke tijdvak is verdeeld in slots. Voor elke slot wordt een \gls{slot_leader} verkozen, die verantwoordelijk is voor het produceren van één blok. Niet alle deelnemers van het netwerk, bijvoorbeeld die minder dan 2\% van de totale circulatie van \glspl{token} hebben, worden geselecteerd om benoemd te worden tot \gls{slot_leader}. Deze groep van deelnemers maken deel uit van de \glspl{elector} groep. \Glspl{elector} verkiezen nieuwe \glspl{slot_leader} gedurende het huidige tijdsvak, waarna er een selectie gemaakt wordt en de nieuwe \glspl{slot_leader} vaststaan voor het volgende tijdsvak. Hoe meer \gls{stake} een deelnemer heeft, hoe groter de kans dat zij uitgekozen wordt om een \gls{slot_leader} te worden in het volgende tijdsvak. De \gls{slot_leader} luistert naar transacties die aangekondigd worden door andere nodes, bundelt ze in een nieuw block, signeert het met zijn private key en publiceert het block in het netwerk \citep{cardano_wiki:proof_of_stake}.

\paragraph{EOS} is een implementatie die gebruik maakt van \acrfull{DPoS} om consensus te bereiken. Het grote verschil tussen \acrshort{DPoS} en \acrshort{PoS}; in een \acrshort{PoS} systeem is elke deelnemer die \gls{stake} heeft maakt onderdeel uitmaken van het validatie- en consensusproces. Met \acrshort{DPoS} kan elke deelnemer die \gls{stake} heeft andere deelnemers verkiezen die onderdeel uitmaken van het validatie- en consensusproces \citep{steemit:eos_dpos}. In contrast met het \acrshort{PoW} algoritme is er geen competitie voor het produceren van een block, maar wordt er samengewerkt om een block te produceren.

\newpage
\section{Gevaren}

Wanneer deelnemers uitmaken van een grootschalig netwerk die niet gecontroleerd wordt door een centrale autoriteit kan het voorkomen dat deelnemers zich misdragen. In juli 2016 is Ethereum opgesplitst in twee partities die dezelfde valuta hanteren; \textit{Ethereum} en \textit{Ethereum Classic}. Dit is veroorzaakt door een kwaadwillende deelnemer in het netwerk die door een bug in het systeem geld naar zichzelf toe kon sturen. Dit heeft ertoe geleid dat veel gebruikers mogelijk een aanzienlijk verlies geleden hebben, waaronder veel ontwikkelaars van Ethereum. Om dit verlies op te lossen werd er een hard-fork voorgesteld die Ethereums code aanpast waarbij de transacties van de kwaadwillende deelnemer teruggedraaid werden \citep{kiffer2017stick}. 

Dit illustreert een van de mogelijke manieren waarop een kwaadwillende gebruiker het systeem kan ondermijnen. Om een duidelijk overzicht te geven van de gevaren binnen een gedecentralizeerd peer-to-peer systeem wordt er onderzocht welke technieken toegepast worden om aanvallen van een kwaadwillende deelnemer van het netwerk tegen te gaan.

\subsection{Eclipse Attack}
Een aanval op het peer-to-peer netwerk waarbij er controle over een deelnemer zijn toegang tot informatie gelimiteerd, of zelfs gemanipuleerd wordt. Met de juiste manipulatie van het peer-to-peer netwerk kan er informatie verduistert worden zodat een goedwillende deelnemer aan het netwerk alleen maar kan communiceren met kwaadwillende deelnemers. Dit kan leiden tot \gls{block_races}, \gls{selfish_mining} en \gls{0-confirmation_double_spending} \citep{heilman2015eclipse}.

\subsection{Majority Attack}
Een aanval waarbij één deelnemer de richting van het netwerk bepaald door het bezitten van 51\% de \gls{voting_power}. In het geval van \acrlong{PoW} betekend dit dat de kwaadwillende deelnemer 51\% van de totale rekenkracht nodig heeft om deze aanval uit te voeren. Dit stelt de kwaadwillende deelnemer in staat om het netwerk te manipuleren en kan leiden tot \gls{0-confirmation_double_spending}.

\subsection{\acrfull{DoS}}
Een algemene benaming voor een collectie van mogelijke oorzaken voor een bewuste verstoring van de services die het peer-to-peer netwerk faciliteert. Dit kan op meerdere manieren optreden, bijvoorbeeld door het invoegen van heel veel transacties in één block, zodat het lang duurt voordat het peer-to-peer netwerk het nieuwe block heeft opgenomen.

\subsection{Sybil Attack}
Een aanval waarbij een deelnemer meerdere virtuele deelnemers creëert in het netwerk waarbij de gecreëerde deelnemers het verkiezingsproces kunnen verstoren door verkeerde informatie door te geven in het netwerk, zoals positief stemmen voor een malafide transactie \citep{conti2017survey}.

\clearpage

\subsection{Double spending}
Bij Creditcard-gebaseerde betalingen wordt er eerlijkheid bereikt door het bestaan van een bank of een andere vertrouwde tussenpersoon (e.g. Paypal). Hierbij wordt de tussenpersoon vertrouwd om te controleren dat diegene die een betaling doet aan een derde partij het geld niet al heeft uitgegeven \citep{karame2012two}. In gedecentralizeerde systemen, waarbij er geen vertrouwe tussenpersoon aanwezig is, staat dit bekend als het \textit{double spending} probleem, waarbij het mogelijk is om \glspl{token} die reeds uitgegeven zijn (i.e. opgenomen in een block) nogmaals gebruikt wordt om een transactie uit te voeren. 

\subsection{Nothing at Stake}
Wanneer er een \gls{fork} ontstaat is de optimale strategie elke replica van de blockchain te valideren, zodat de diegene die het validatie proces uitvoert nog steeds uitbetaald krijgt, ongeacht of de \gls{fork} geaccepteerd wordt of niet.

\newpage

\section{Identiteit}

Blockchain kan een zeker mate van privacy garanderen door de public en private keys, wat ervoor zorgt dat een gebruiker niet zijn echte identiteit hoeft te hanteren om met het systeem te interacteren. Echter, \cite{meiklejohn2013fistful} toont aan dat blockchain niet de transactionele privacy kan waarborgen omdat de waarden van alle transacties en saldo van elke public key openbaar inzichtbaar zijn.

\citet{Okamoto:1991:UEC:646756.705374} beschrijft zes criteria waaraan de ideale implementatie van elektronisch geld moet voldoen. In het bijzonder worden er twee criteria genoemd:

\begin{itemize}
  \item \textbf{Untraceability:} voor elke inkomende transactie hebben alle mogelijke afzenders gelijke kansen om geïdentificeerd te worden als verstuurder. 
  
  \item \textbf{Unlinkability:} voor elke twee uitgaande transacties moet het onmogelijk zijn om aan te tonen dat ze naar dezelfde persoon verstuurd zijn.
\end{itemize}

\newpage

\section{Bitcoin}
\subsection{Functionaliteit}
\paragraph{Majority Attack}
\cite{nakamoto2008bitcoin} stelt dat het uitvoeren van een majority attack op het netwerk onpraktisch is omdat het uitvoeren ervan niet opweegt tegen de kosten voor de benodigde hardware om de rekenkracht te behalen die hiervoor nodig is. Dit blijkt niet altijd het geval, \cite{eyal2014majority} beschrijft namelijk een strategie genaamd Selfish Mining waarbij er gevalideerde blocks achtergehouden worden voor het netwerk waardoor er opzettelijk een fork wordt gecreëerd. De eerlijke miners zullen verder werken aan de publiekelijke blockchain terwijl de uitvoerder van het Selfish Mining strategie verder werkt op de achtergehouden blockchain. Als de uitvoerder meer blokken ontdekt onstaat er een voorsprong op de publiekelijke blockchain en worden de blocks nog steeds achtergehouden. Wanneer de lengte van de publiekelijke blockchain de lengte van de achtergehouden blockchain benaderd, zal de uitvoerder de blockchain publiceren. Dit leid ertoe dat miners die het Bitcoin protocol volgen hun middelen verspillen aan het minen van cryptopuzzles die er niet toe doen.

\paragraph{Denial of Service} 
Over de jaren heen zijn er kwetsbaarheden in het Bitcoin protocol geïdentificeerd die het mogelijk maken om een \acrshort{DoS} aanval uit te voeren. De meest recente\footnote{Er zijn recentere aanvallen op het Bitcoin protocol geweest waarbij er \acrshort{DoS} aanval heeft plaatsgevonden maar deze zijn niet nader gespecificeerd, zie: \href{https://en.bitcoin.it/wiki/Common_Vulnerabilities_and_Exposures}{"Common Vulnerabilities and Exposures - Bitcoin Wiki"}.} aanval {\citep{nist_bitcoind_dos} exploiteert een zwakheid in de implementatie van een \Gls{bloom_filter}, een filter die onder andere gebruikt wordt door \glspl{wallet_node} om alleen transacties binnen te halen waarbij de deelnemer betrokken is. Hierdoor was het mogelijk om een sequentie van berichten te sturen die ervoor zorgde dat een volledige node binnen het netwerk overbelast werd.

\paragraph{Eclipse Attack} 
\citeauthor{heilman2015eclipse} heeft aangetoond dat Bitcoin's peer discovery mechanisme toegankelijk is voor een \textit{Eclipse attack}. Door de manier waarop het Peer Discovery mechanisme werkt is het mogelijk om de lijst van connecties zo te manipuleren dat nieuwe deelnemers doorgestuurd worden naar kwaadwillende deelnemers.

\paragraph{Double spending}
\citeauthor{karame2012double} toont aan dat het in het beginstadium van het Bitcoin protocol mogelijk was om via zogenaamde 'fast payments' een double spending aanval uit te voeren. 

\subsection{Gevaren}
\paragraph{Majority Attack}
\cite{nakamoto2008bitcoin} stelt dat het uitvoeren van een majority attack op het netwerk onpraktisch is omdat het uitvoeren ervan niet opweegt tegen de kosten voor de benodigde hardware om de rekenkracht te behalen die hiervoor nodig is. Dit blijkt niet altijd het geval, \cite{eyal2014majority} beschrijft namelijk een strategie genaamd Selfish Mining waarbij er gevalideerde blocks achtergehouden worden voor het netwerk waardoor er opzettelijk een fork wordt gecreëerd. De eerlijke miners zullen verder werken aan de publiekelijke blockchain terwijl de uitvoerder van het Selfish Mining strategie verder werkt op de achtergehouden blockchain. Als de uitvoerder meer blokken ontdekt onstaat er een voorsprong op de publiekelijke blockchain en worden de blocks nog steeds achtergehouden. Wanneer de lengte van de publiekelijke blockchain de lengte van de achtergehouden blockchain benaderd, zal de uitvoerder de blockchain publiceren. Dit leid ertoe dat miners die het Bitcoin protocol volgen hun middelen verspillen aan het minen van cryptopuzzles die er niet toe doen.

\paragraph{Denial of Service} 
Over de jaren heen zijn er kwetsbaarheden in het Bitcoin protocol geïdentificeerd die het mogelijk maken om een \acrshort{DoS} aanval uit te voeren. De meest recente\footnote{Er zijn recentere aanvallen op het Bitcoin protocol geweest waarbij er \acrshort{DoS} aanval heeft plaatsgevonden maar deze zijn niet nader gespecificeerd, zie: \href{https://en.bitcoin.it/wiki/Common_Vulnerabilities_and_Exposures}{"Common Vulnerabilities and Exposures - Bitcoin Wiki"}.} aanval {\citep{nist_bitcoind_dos} exploiteert een zwakheid in de implementatie van een \Gls{bloom_filter}, een filter die onder andere gebruikt wordt door \glspl{wallet_node} om alleen transacties binnen te halen waarbij de deelnemer betrokken is. Hierdoor was het mogelijk om een sequentie van berichten te sturen die ervoor zorgde dat een volledige node binnen het netwerk overbelast werd.

\paragraph{Eclipse Attack} 
\citeauthor{heilman2015eclipse} heeft aangetoond dat Bitcoin's peer discovery mechanisme toegankelijk is voor een \textit{Eclipse attack}. Door de manier waarop het Peer Discovery mechanisme werkt is het mogelijk om de lijst van connecties zo te manipuleren dat nieuwe deelnemers doorgestuurd worden naar kwaadwillende deelnemers.

\paragraph{Double spending}
\citeauthor{karame2012double} toont aan dat het in het beginstadium van het Bitcoin protocol mogelijk was om via zogenaamde 'fast payments' een double spending aanval uit te voeren. 

\subsection{Identiteit}
\paragraph{Majority Attack}
\cite{nakamoto2008bitcoin} stelt dat het uitvoeren van een majority attack op het netwerk onpraktisch is omdat het uitvoeren ervan niet opweegt tegen de kosten voor de benodigde hardware om de rekenkracht te behalen die hiervoor nodig is. Dit blijkt niet altijd het geval, \cite{eyal2014majority} beschrijft namelijk een strategie genaamd Selfish Mining waarbij er gevalideerde blocks achtergehouden worden voor het netwerk waardoor er opzettelijk een fork wordt gecreëerd. De eerlijke miners zullen verder werken aan de publiekelijke blockchain terwijl de uitvoerder van het Selfish Mining strategie verder werkt op de achtergehouden blockchain. Als de uitvoerder meer blokken ontdekt onstaat er een voorsprong op de publiekelijke blockchain en worden de blocks nog steeds achtergehouden. Wanneer de lengte van de publiekelijke blockchain de lengte van de achtergehouden blockchain benaderd, zal de uitvoerder de blockchain publiceren. Dit leid ertoe dat miners die het Bitcoin protocol volgen hun middelen verspillen aan het minen van cryptopuzzles die er niet toe doen.

\paragraph{Denial of Service} 
Over de jaren heen zijn er kwetsbaarheden in het Bitcoin protocol geïdentificeerd die het mogelijk maken om een \acrshort{DoS} aanval uit te voeren. De meest recente\footnote{Er zijn recentere aanvallen op het Bitcoin protocol geweest waarbij er \acrshort{DoS} aanval heeft plaatsgevonden maar deze zijn niet nader gespecificeerd, zie: \href{https://en.bitcoin.it/wiki/Common_Vulnerabilities_and_Exposures}{"Common Vulnerabilities and Exposures - Bitcoin Wiki"}.} aanval {\citep{nist_bitcoind_dos} exploiteert een zwakheid in de implementatie van een \Gls{bloom_filter}, een filter die onder andere gebruikt wordt door \glspl{wallet_node} om alleen transacties binnen te halen waarbij de deelnemer betrokken is. Hierdoor was het mogelijk om een sequentie van berichten te sturen die ervoor zorgde dat een volledige node binnen het netwerk overbelast werd.

\paragraph{Eclipse Attack} 
\citeauthor{heilman2015eclipse} heeft aangetoond dat Bitcoin's peer discovery mechanisme toegankelijk is voor een \textit{Eclipse attack}. Door de manier waarop het Peer Discovery mechanisme werkt is het mogelijk om de lijst van connecties zo te manipuleren dat nieuwe deelnemers doorgestuurd worden naar kwaadwillende deelnemers.

\paragraph{Double spending}
\citeauthor{karame2012double} toont aan dat het in het beginstadium van het Bitcoin protocol mogelijk was om via zogenaamde 'fast payments' een double spending aanval uit te voeren. 


\newpage

\section{Cardano}
\subsection{Functionaliteit}
De cardano implementatie is een public Blockchain waarbij alle transacties inzichtbaar zijn en iedereen mee kan doen aan het consensus proces. Het maakt gebruik van public- en private keys om pseudonimiteit te waarborgen, waarbij de elliptic curve cryptografie implementatie Curve25519 toegepast wordt om de public- en private key te genereren. Binnen Cardano worden er verschillende adressen gebruikt om transacties van een bestemming te voorzien \citep[''Addresses in Cardano SL'']{cardano_wiki}:

\begin{enumerate}
  \item \textbf{public key address}
  \\ Een base58 gecodeerde string van de public key dat gebruikt wordt als bestemming van een transactie.
  \item \textbf{script address}
  \\ Wordt gebruikt voor het Pay to Script Hash principe, waarbij er in plaats van de public key gebruikt wordt als bestemming, een validatie script verstuurd wordt die gebruik maakt van een zogenaamde redemption script. Om de waarde van de transactie te claimen dient het validatie script positief uit te vallen.
  \item \textbf{redeem address}
  \\ Wordt gebruikt voor het Pay to Public Key Hash principe, waarbij er een hash gecreëerd wordt wat ervoor zorgt dat de public key alleen publiekelijk geregistreerd wordt wanneer de output van een transactie wordt uitgegeven.
\end{enumerate}




\subsection{Gevaren}
De cardano implementatie is een public Blockchain waarbij alle transacties inzichtbaar zijn en iedereen mee kan doen aan het consensus proces. Het maakt gebruik van public- en private keys om pseudonimiteit te waarborgen, waarbij de elliptic curve cryptografie implementatie Curve25519 toegepast wordt om de public- en private key te genereren. Binnen Cardano worden er verschillende adressen gebruikt om transacties van een bestemming te voorzien \citep[''Addresses in Cardano SL'']{cardano_wiki}:

\begin{enumerate}
  \item \textbf{public key address}
  \\ Een base58 gecodeerde string van de public key dat gebruikt wordt als bestemming van een transactie.
  \item \textbf{script address}
  \\ Wordt gebruikt voor het Pay to Script Hash principe, waarbij er in plaats van de public key gebruikt wordt als bestemming, een validatie script verstuurd wordt die gebruik maakt van een zogenaamde redemption script. Om de waarde van de transactie te claimen dient het validatie script positief uit te vallen.
  \item \textbf{redeem address}
  \\ Wordt gebruikt voor het Pay to Public Key Hash principe, waarbij er een hash gecreëerd wordt wat ervoor zorgt dat de public key alleen publiekelijk geregistreerd wordt wanneer de output van een transactie wordt uitgegeven.
\end{enumerate}




\subsection{Identiteit}
De cardano implementatie is een public Blockchain waarbij alle transacties inzichtbaar zijn en iedereen mee kan doen aan het consensus proces. Het maakt gebruik van public- en private keys om pseudonimiteit te waarborgen, waarbij de elliptic curve cryptografie implementatie Curve25519 toegepast wordt om de public- en private key te genereren. Binnen Cardano worden er verschillende adressen gebruikt om transacties van een bestemming te voorzien \citep[''Addresses in Cardano SL'']{cardano_wiki}:

\begin{enumerate}
  \item \textbf{public key address}
  \\ Een base58 gecodeerde string van de public key dat gebruikt wordt als bestemming van een transactie.
  \item \textbf{script address}
  \\ Wordt gebruikt voor het Pay to Script Hash principe, waarbij er in plaats van de public key gebruikt wordt als bestemming, een validatie script verstuurd wordt die gebruik maakt van een zogenaamde redemption script. Om de waarde van de transactie te claimen dient het validatie script positief uit te vallen.
  \item \textbf{redeem address}
  \\ Wordt gebruikt voor het Pay to Public Key Hash principe, waarbij er een hash gecreëerd wordt wat ervoor zorgt dat de public key alleen publiekelijk geregistreerd wordt wanneer de output van een transactie wordt uitgegeven.
\end{enumerate}





\newpage

\section{EOS}
\subsection{Functionaliteit}
EOS is een consortium Blockchain waarin de identiteit van een gebruiker vastgelegd wordt in een account model, waarbij een account identificeerbaar is door een unieke naam van maximaal twaalf karakters. Handeling zijn geresticteerd door middel van een Role Based Permissie systeem. Om dit mogelijk te maken dient een gebruiker allereerst geautoriseerd te zijn alvorens deel te kunnen nemen aan het netwerk. Centraal in de implementatie staat de notie van Actions \& Handlers. Elk account (i.e.\ deelnemer) heeft een eigen database die alleen toegankelijk is door gedefinieerde action handlers. Dit systeem is soortgelijk aan smart contracts zoals in gebruik bij Ethereum.


\subsection{Gevaren}
EOS is een consortium Blockchain waarin de identiteit van een gebruiker vastgelegd wordt in een account model, waarbij een account identificeerbaar is door een unieke naam van maximaal twaalf karakters. Handeling zijn geresticteerd door middel van een Role Based Permissie systeem. Om dit mogelijk te maken dient een gebruiker allereerst geautoriseerd te zijn alvorens deel te kunnen nemen aan het netwerk. Centraal in de implementatie staat de notie van Actions \& Handlers. Elk account (i.e.\ deelnemer) heeft een eigen database die alleen toegankelijk is door gedefinieerde action handlers. Dit systeem is soortgelijk aan smart contracts zoals in gebruik bij Ethereum.


\subsection{Identiteit}
EOS is een consortium Blockchain waarin de identiteit van een gebruiker vastgelegd wordt in een account model, waarbij een account identificeerbaar is door een unieke naam van maximaal twaalf karakters. Handeling zijn geresticteerd door middel van een Role Based Permissie systeem. Om dit mogelijk te maken dient een gebruiker allereerst geautoriseerd te zijn alvorens deel te kunnen nemen aan het netwerk. Centraal in de implementatie staat de notie van Actions \& Handlers. Elk account (i.e.\ deelnemer) heeft een eigen database die alleen toegankelijk is door gedefinieerde action handlers. Dit systeem is soortgelijk aan smart contracts zoals in gebruik bij Ethereum.



\section{Monero}
\subsection{Functionaliteit}
\paragraph{Double Spending} door gebruik te maken van \glspl{ring_signature} wordt de herkomst van een transactie gemaskeerd door de handtekening van de verstuurder te groeperen met handtekeningen vanuit outputs die reeds gedaan zijn in de Blockchain. Een probleem dat hierbij optreed is de mogelijkheid tot de uitvoering van \gls{double_spending} omdat een transactie lastiger is te valideren. Hierdoor maakt Monero gebruik van \gls{key_image}. Een \gls{key_image} wordt gebruikt om te valideren dat de private key die gebruikt is om de transactie te ondertekenen niet eerder gebruikt is, zonder te onthullen welke handtekening het is.

\subsection{Gevaren}
\paragraph{Double Spending} door gebruik te maken van \glspl{ring_signature} wordt de herkomst van een transactie gemaskeerd door de handtekening van de verstuurder te groeperen met handtekeningen vanuit outputs die reeds gedaan zijn in de Blockchain. Een probleem dat hierbij optreed is de mogelijkheid tot de uitvoering van \gls{double_spending} omdat een transactie lastiger is te valideren. Hierdoor maakt Monero gebruik van \gls{key_image}. Een \gls{key_image} wordt gebruikt om te valideren dat de private key die gebruikt is om de transactie te ondertekenen niet eerder gebruikt is, zonder te onthullen welke handtekening het is.

\subsection{Identiteit}
\paragraph{Double Spending} door gebruik te maken van \glspl{ring_signature} wordt de herkomst van een transactie gemaskeerd door de handtekening van de verstuurder te groeperen met handtekeningen vanuit outputs die reeds gedaan zijn in de Blockchain. Een probleem dat hierbij optreed is de mogelijkheid tot de uitvoering van \gls{double_spending} omdat een transactie lastiger is te valideren. Hierdoor maakt Monero gebruik van \gls{key_image}. Een \gls{key_image} wordt gebruikt om te valideren dat de private key die gebruikt is om de transactie te ondertekenen niet eerder gebruikt is, zonder te onthullen welke handtekening het is.
