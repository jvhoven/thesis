\newpage
\chapter{Opzet}

Het onderzoek dient voor het opstellen van het adviesrapport waarin protocollen worden  gepresenteerd  aan Quintor die mogelijk geïmplementeerd kunnen worden tijdens de realisatie van het Proof-of-Concept. Het betreft exploratief onderzoek waarin case-study gebruikt wordt om een gedetailleerde omschrijving van de onderdelen Identity Management en Distributed Network op op te stellen van de Blockchain implementaties Bitcoin, Cardano, EOS en Monero. De kennis die hiermee wordt opgebouwd kan eventueel gebruikt worden in vervolgonderzoek.

In het onderzoek staat de onderstaande hoofdvraag centraal:
\begin{formal}
  Welke protocol implementaties kunnen toegepast worden om de onderdelen Distributed Network en Identity Management te realiseren voor een Blockchain implementatie?
\end{formal}

Omdat de hoofdvraag te groot is om in een keer te beantwoorden is het opgesplitst in de volgende deelvragen:

\begin{enumerate}[noitemsep]
  \item "Welke soorten gedistribueerde netwerken worden er gebruikt?"
  \item "Hoe werken de gedistribueerde netwerken en tegen welke gevaren zijn ze bestendig?"
  \item "Hoe wordt er omgegaan met de identiteit van de gebruiker?
\end{enumerate}
