\newpage
\chapter{Conclusie}

In dit onderzoek is er gezocht naar een antwoord op de vraag: ``Welke protocol implementaties kunnen toegepast worden om de onderdelen Distributed Network en Identity Management te realiseren voor een Blockchain implementatie?''. Hiervoor is kwalitatief onderzoek uitgevoerd naar de Blockchain implementaties, EOS, Cardano, Monero en Bitcoin.

\section{Deelvragen}

\begin{enumerate}
  \item \textbf{``Welke soorten gedistribueerde netwerken worden er gebruikt in de implementaties?''}
  \\ Een gedistribueerd netwerk binnen Blockchain is getypeerd aan het consensus protocol dat gebruikt wordt. In het onderzoek zijn er twee soorten geïdentificeerd, netwerken die gebruik maken van Proof of Stake of van Proof of Work.
  
  \newpage
  \item \textbf{``Hoe werken de gedistribueerde netwerken van de implementaties en tegen welke gevaren zijn ze bestendig?''}
  \\ In het onderzoek is de functionaliteit beschreven die ondersteund wordt door een gedistribueerde netwerk van een implementatie en is er aandacht besteed aan de oplossingen die het netwerk gebruikt om aanvallen tegen te gaan. 
  \begin{itemize}
    \item \textbf{Bitcoin}
    \\ Het netwerk van Bitcoin communiceert via TCP/IP en maakt gebruik van bootstrap nodes waarmee connectie wordt gemaakt op het moment dat een nieuwe deelnemer het netwerk wilt toetreden. Informatie wordt verstuurd door een voorafgedefinieerde set aan berichttypes: \textit{inv}, \textit{tx}, \textit{block}, \textit{getdata}, waarbij een \textit{inv} bericht gebruikt wordt ter inventarisatie over de beschikbaarheid van data, \textit{tx} bericht om een transactie te versturen, \textit{block} bericht om een block te versturen, \textit{getdata} bericht om data op te vragen. \\ \\ Op het Bitcoin netwerk zijn meerdere aanvallen in de loop der jaren uitgevoerd en geïdentificeerd, een studie uit 2015 gedaan door \cite{heilman2015eclipse} toont aan dat het Peer Discovery mechanisme vatbaar is voor een Sybil Attack. \cite{nakamoto2008bitcoin} stelt dat de voordelen van het uitvoeren van een majority attack niet opweegt tegen de kosten voor de benodigde hardware om de rekenkracht te behalen. \cite{eyal2014majority} beschrijft dat het niet nodig is om een merendeel van de rekenkracht te bezitten en introduceert de aanval \gls{selfish_mining}.

    \item \textbf{Cardano}
    \\ Het netwerk van Cardano communiceert via TCP/IP en maakt gebruik van het Kademlia protocol waardoor het maar nodig is om één bootstrap node te gebruiken om het netwerk toe te treden. De achterliggende structuur van Kademlia is een Binary Tree waarbij de positie van een deelnemer in de Binary Tree bepaald wordt door een unieke prefix van de identificatiecode. Het protocol garandeert dat een deelnemer in verbinding staat met ten minste één andere deelnemer. Informatie wordt uitgewisseld door drie abstracte berichttypes: \textit{inv}, \textit{req}, en \textit{data}. Het \textit{inv} bericht wordt gebruikt om aan te geven dat er data beschikbaar is, het \textit{req} bericht wordt gebruikt om beschikbare data op te vragen en het \textit{data} bericht wordt vervolgens gebruikt om de data te versturen. \\ \\ Implementaties die gebruik maken van \acrshort{PoS} zijn afhankelijk van de manier waarop een leiderschapsverkiezing wordt gesimuleerd, waarbij er grote kans is dat het gevoelig is voor beïnvloedingen van kwaadwillende deelnemers in het netwerk in de vorm van een Sybil Attack. Cardano heeft een zwak punt in het Kademlia netwerk geïdentificeerd waardoor het mogelijk zou zijn om Eclipse Attack uit te voeren.

    \item \textbf{Monero}
    \\ Het netwerk van Monero maakt gebruik van het \acrfull{I2P} protocol, dat zowel UDP/IP als TCP/IP ondersteund. Om het netwerk toe te treden wordt er gebruik gemaakt van bootstrap nodes die vastgelegd zijn in de broncode. Communicatie wordt gedaan door middel van \Glspl{tunnel}, waarbij elke deelnemer twee \Glspl{tunnel}, een inkomende en een uitgaande, heeft voor elke connectie.

    \item \textbf{EOS}
    \\ \textit{Ten tijde van het onderzoek is er geen technische beschrijving beschikbaar over het netwerk component van EOS.}
  \end{itemize}

  \newpage
  \item \textbf{``Hoe wordt er omgegaan met de identiteit van de gebruiker binnen de implementatie?''}

  \begin{itemize}
    \item \textbf{Bitcoin}
    \\ Bitcoin is een public Blockchain waarbij de gehele historie van transacties publiekelijk in te zien is. Een deelnemer in het Bitcoin netwerk wordt geïdentificeerd aan de hand van zijn public key. Deze public key wordt onder andere opgenomen in transacties om de betaler en de ontvanger te registreren. In een studie gedaan door \cite{reid2013analysis} wordt er een analyse model opgezet dat aantoont dat het Bitcoin protocol niet aan de untraceability eis voldoet.

    \item \textbf{Cardano}
    \\ Cardano is een public Blockchain waarbij de gehele historie van transacties publiekelijk in te zien is. Cardano maakt gebruik van public- en private key cryptografie om pseudonimiteit te waarborgen. Deze keys worden gebruikt om een transactie van een bestemming te voorzien, waarbij er drie definities van adressen gebruikt worden: een public key address, een script address en een redeem address.

    \item \textbf{EOS}
    \\ EOS is een consortium Blockchain waarbij gebruikers zichzelf identificeren met een unieke naam van maximaal twaalf karakters. Om te participeren binnen het netwerk dient er toegang verleent te worden door een authenticatie proces alvorens de deelnemer wordt toegelaten.  Handeling binnen het netwerk worden gevalideerd door een Role Based Permissie systeem, waarbij permissies gekoppeld zijn aan actions die vastgelegd zijn in de lokale database.

    \item \textbf{Monero}
    \\ Monero is een public Blockchain waarbij de gehele historie van transacties publiekelijk in te zien is. Binnen Monero heeft elke deelnemer een account die gebaseerd is op twee keys: \gls{spend_key} en een \gls{view_key}. Door het afleiden van een eenmalige public key, ook wel een \gls{stealth_address} genoemd, uit de \gls{spend_key} en \gls{view_key} garandeert het Monero protocol unlinkability. Untraceability wordt behaald door het gebruik van \glspl{ring_signature}. Hierbij worden meerdere \glspl{stealth_address} toegevoegd aan een transactie, waarbij een afkomstig van de verstuurder van de transactie en de rest aangevuld door eerder gebruikte \glspl{stealth_address} in de Blockchain. Hierdoor wordt de herkomst van een transactie gemaskeerd.
  \end{itemize}
\end{enumerate}

\newpage
\section{Hoofdvraag}
Tezamen beantwoorden de deelvragen de hoofdvraag:

\begin{formal}
  Welke protocol implementaties kunnen toegepast worden om de onderdelen Distributed Network en Identity Management te realiseren voor een Blockchain implementatie?
\end{formal}

\subsection{Distributed Network}

Zowel het gedistribueerd netwerk van Monero, Cardano (Kademlia) en Bitcoin maken gebruik van bootstrap nodes om een deelnemer toe te laten treden, waarbij het Kademlia protocol kan functioneren met één bootstrap node.

\begin{tabular}{|p{5cm}|p{10cm}|}
  \hline
  \textbf{Protocol} & \textbf{Toelichting} \\
  \hline
  Kademlia & Een bestaand protocol gerealiseerd door \cite{maymounkov2002kademlia}. Dit protocol heeft een aantal wijzigingen binnen Cardano, zoals het versturen van informatie gaat over TCP/IP en er is een uitbreiding gemaakt op de manier waarop identificatiecodes toegekend worden aan deelnemers om een mogelijke \gls{sybil_attack} uit te sluiten. \\
  \hline
  Bitcoin & Communicatie verloopt over TCP/IP waarbij informatie wordt verstuurd door \textit{inv}, \textit{tx}, \textit{block} en \textit{getdata} berichten. \\
  \hline
  Monero & Focust op de privacy van de gebruiker en maakt gebruik van \acrfull{I2P} om deze anonimiteit binnen het netwerk te waarborgen. \\
  \hline
\end{tabular}

\subsection{Identity Management}

\begin{tabular}{|p{5cm}|p{10cm}|}
  \hline
  \textbf{Protocol} & \textbf{Toelichting} \\
  \hline
  Bitcoin & Maakt gebruik van het UTXO-model, waarin public- en private keys gebruikt worden om de betaler en ontvanger te registreren binnen een transactie. Door het gebruik van het analysemodel gepresenteerd door \cite{reid2013analysis} is aangetoond dat het Bitcoin niet aan de niet aan de untraceability en unlinkability eis voldoet. \\
  \hline
  Cardano & Maakt gebruik van het UTXO-model, waarin public- en key cryptografie gebruikt wordt. Er is hierbij geen studie gevonden die aantoont dat het voldoet aan de untraceability en unlinkability eis, maar heeft aanzienlijke overeenkomsten met hoe Bitcoin omgaat met de identiteit. \\
  \hline
  EOS & Maakt gebruikt van het Account-model, waarin een gebruiker een unieke naam van maximaal twaalf karakters hanteert als identiteit. Daarnaast hanteert EOS de volgende componenten:
  \begin{enumerate}
    \item Role Based Permission Management
    \item Actions \& Handlers
  \end{enumerate}
  \\
  \hline
\end{tabular}